
%*********************************************************
% FILE NAME: instruct.tex
% Copyright: 20jun04
% CREATED BY: Carl de Boor and Allan Pinkus, based on the Nashboro Press file
% EMAIL: deboor@cs.wisc.edu, pinkus@tx.technion.ac.il
% WEB: www.cs.wisc.edu/~deboor/, ???
% ******************************************************************
%
%                For use in preparing papers for
%
%                Surveys in Approximation Theory

%******************************************************

\def\updated{03jun06}
   %% supply today's date, to help keep track of versions of this paper

\count100= 1
\count101= 9
\input satmacros.tex

\overfullrule = 0 pt

% \singlecount   %% turn this on to use just one numbering system.
\def\presect{} %% turn this on to make numbering independent of sections.

% \draft %% turn this on to see, in the margin, the label names.

\title{Instructions for Using the||SAT TEX Macros}
\author{A.~N.~Onymous}

\def\shorttitle{Instructions}
\def\shortauthor{A.~N.~Onymous}
\def\mscnumbers{41-02}
\def\keywords{words to the wise}

%   ********************** MACROS  ******************************

\def\backslash{{\tt \char'134}}
\def\co{{\cal O}}
\def\cP{{\cal P}}
\def\ct{{\cal T}}
\def\ctil{\tilde{\cal T}}
\def\lb{\char'173}
\def\rb{\char'175}
\def\oln{\co(n \log(n))}
\def\tb#1{{\tt \backslash #1}}
\def\Tex{{\TeX~}}

%************************* ABSTRACT **********************************

\abstract {A good abstract consists of a single paragraph
describing the main results of the paper. You may use mathematics
in the abstract, but it is better not to put too many symbols into
the abstract.}

%******************************* BODY *********************************

\sect{Introduction}
We would like all authors to follow some set of standards for
layout and style.  The purpose of this document is to explain how
to use the plain \Tex macro file {\tt satmacros.tex} to prepare
your contribution.  One way to learn how to use these macros is to look at
the \Tex file {\tt instruct.tex} which was used to generate this document.
For those in a hurry, just obtain a copy of the file {\tt textemplate.tex}
from the page
{\tt http://www.math.technion.ac.il/sat/authors.html}, rename it, and fill in
the details.

If you are a La\Tex user, simply follow the instructions embedded in the 
La\Tex template {\tt latextemplate.tex} obtainable from
{\tt http://www.math.technion.ac.il/sat/authors.html}.

\sect{Organization} We recommend that you organize your paper into
sections. These sections will be automatically numbered. To create
section headings, use the \tb{sect} macro.  If your section
heading does not fit on one line, use {\tt ||} to indicate
suitable linebreaks in the heading. Do not leave a blank line
between the section heading and the first line of text. A blank
line indents the first paragraph in the section. Each major word
in the section heading (other than words like {\it a, and, the,
to, with}, etc.) should be capitalized. Do not include a period at
the end of the section heading. Subsections can be created with
\tb{subsect} and \tb{fsubsect}. The latter is for use immediately
after the start of a section and simply leaves a little less
space.

The command \tb{remember}{\tt\lb}\tb{the}\tb{sectionnum}{\tt\rb{}\lb...\rb}
remembers the current section number in such a way that you can refer to it
at any point by the command \tb{recall}{\tt\lb...\rb}.

\sect{Proclamations} Definitions, Lemmas, Theorems, and
Propositions are examples of proclamations.  You create them using
the macro \tb{proclaim}.  This sets the title in boldface and the
body of the object in a slanted font as, for example, in the following.

\proclaim Definition \label{def1}. Let $\cP$ be a finite subset of $\RR^2$.
A subset ${\cal T}$ of $\cP^3$ is called a
\dword{triangulation} of $\cP$ provided that
\ritem{(1)} for each $t\in {\cal T}$, the points $t_1, t_2, t_3$ are the
vertices of a proper triangle, denoted $T_t$;
\ritem{(2)} each such triangle $T_t$ contains no points from $\cP$ other than
its vertices.
\nopf

\proclaim Theorem \label{thm2}.  Given any point set $\cal P$, there always
exists at least one optimal triangulation.  Moreover, every optimal
triangulation is locally optimal provided
$$ n = \cases{3,& if $m$ is odd, \cr
              0,& otherwise. \cr}$$

\pf
The \tb{proclaim} macro was used to state both the definition
and the theorem.
If you are not giving a proof, end your theorem with \tb{nopf},
which at the moment is just a {\tt medskip}.
Otherwise, leave a blank line and use
\tb{pf} to begin the proof.  In stating this theorem, we
have illustrated the use of the macro \tb{cases}.
The end of the proof is marked with a square box by using the
\tb{eop} macro. If the proof ends with a displayed equation,
put \tb{meop} at the end of the equation. \eop

\sect{Lists}
The usual \Tex macro for creating lists is called \tb{item}.
However, if you are creating lists within a proclamation,
as was done in Def.~\recall{def1} above, it looks better to use the
macro \tb{ritem} which is defined in {\tt satmacros.tex}.
It sets the numbers in roman rather than slant. Note that
we prefer that you number such lists as (1), (2), etc.
rather than as 1., 2., etc.
If you don't want to number the items in a list, you can
use our macro \tb{bull} in place of \tb{item} or \tb{ritem}.
For an example, see the lists in Sects.~\recall{nottodo} and \recall{refs}
below.

\sect{Open Face Letters} Use the macros \tb{RR} and \tb{CC} for
the sets $\RR$ and $\CC$ of real and complex numbers. You may also
want to use the macros \tb{NN}, \tb{PP}, and \tb{ZZ} to get $\NN$,
$\PP$,  and $\ZZ$. If you have to use $\RR$ in the title, use the
macro \tb{RRt}. You may also use the macros \tb{Cbb}, \tb{Dbb},
\tb{Kbb}, \tb{Nbb}, \tb{Pbb}, \tb{Qbb}, \tb{Rbb}, \tb{Tbb}, and
\tb{Zbb}, to get $\Cbb, \Dbb, \Kbb,\Nbb,\Pbb,\Qbb,\Rbb,\Tbb,\Zbb$.

\sect{Bold Face Characters} To create bold-faced letters, you can
use \tb{bf}. Thus, to get $\bf{n}$ you type \tb{bf}{\tt\lb{}n\rb}.
\TeX{} does not include a way to get bold face
mathematical symbols. For that reason, we have included in {\tt
satmacros.tex} the macros \tb{bfm}, as well as \tb{bfs} and \tb{bfss}, the
latter for subscripts and subsubscripts, respectively, as in
$$\beta_{\beta_\beta}\;\not=\;\bfm{\beta}_{\bfs{\beta}_{\bfss{\beta}}}.$$

\sect{Displayed Equations} Equations can be numbered or not
numbered as you like.  However, if you number them, we want equation
numbers to appear on the right. The appropriate macro is
\tb{eqno}. For short papers, we suggest numbering sequentially
starting with (1).

If a sentence ends within a displayed formula, the period should
be inserted in the formula after the last symbol. If the sentence
continues, but a comma is appropriate at the end of the displayed
formula, put it there.  {\it Do not} insert any extra space in
front of such punctuation.

To line up displayed equations you can use \tb{eqalign}
or \tb{eqalignno}. Here is an example
$$\eqalign{N& =  2n-n_b-2, \qquad 1 \le n \le 16, \cr
            e & =  3n-n_b-3, \qquad 1 \le n \le 4. \cr}\eqno(1)$$
The idea is simple -- you use
the symbol $\&$ to mark the alignment points, and you end each
line with \tb{cr}.

To line up equations while
giving each of them its own number, you use \tb{eqalignno}.
Here is an example:
$$\eqalignno{N& =  2n-n_b-2, \qquad 1 \le n \le 16, & (2)\cr
            e & =  3n-n_b-3, \qquad 1 \le n \le 4.& (3) \cr}$$
% remember the current equation number for later use in the discussion of
% automatic numbering.
\eqnum3

\sect{Figures}
Almost everyone these days creates figures in postscript.  These
can be integrated into your manuscript with the aid of the macro
\tb{placefig} to which you supply the name of the postscript
file and the width, in truecm, to which your figure should be scaled.
Use the macro \tb{figcaption} to supply the caption for your figure(s),
using {\tt ||} to indicate line breaks in case the caption is longer than
one line.

As an example, the commands

\leftline{\tb{placefig}{\tt\lb instruct.fig\rb\lb{5}\rb}}
\leftline{\tb{figcaption}{\tt\lb\lb}\tb{bf}{\tt\ Figure 1.\rb\  The
function \$f(x) = (x-t)$\underline{\ }$+\$.\rb}} result in the following
picture:
\placefig{instruct.fig}{5}
\figcaption{{\bf Figure 1.\ } The function $f(x) = (x-t)_+$.}

No matter how you generate figures, and regardless of whether they
are {\tt ps} files or {\tt eps} files, it would help greatly if you would name
your figure files in the form {\tt name1.fig}, {\tt name2a.fig},
{\tt name2b.fig}, etc., where {\tt name} is the name of the first
author, or some abbreviation of it.

Wouldn't it be nice if you could label your figures using exactly the same
fonts as are used in the rest of your paper? You can, with the aid of the
macros \tb{figinbox} and \tb{ppoint} from {\tt satmacros.tex}. For example,
the next figure is generated by the following commands:


\leftline{\tb{showgridfalse}}
\leftline{\tb{figinbox} {\tt instruct.fig(3.1,5.4)\lb{5}\rb\lb}}
\leftline{\tb{ppoint}{\tt01(2.8,2.1)\lb{\$t\$}\rb}}
\leftline{\tb{ppoint}{\tt10(2.5,5.1)\lb{\$x\$}\rb\rb}}
\leftline{\tb{figcaption}{\tt\lb\lb}\tb{bf}{\tt\ Figure 2.\rb\  The
function \$f(x) = (x-t)$\underline{\ }$+\$.||(Isn't}}
\leftline{\tt it more informative with those labels?)\rb}

\showgridfalse
\figinbox instruct.fig(3.1,5.4){5}{
\ppoint01(2.8,2.1){$t$}
\ppoint10(2.5,5.1){$x$}}
\figcaption{{\bf Figure 2.\ } The
function $f(x) = (x-t)_+$.||(Isn't it more informative with those labels?)}

The placement macro \tb{ppoint}{\tt ij(d,r)\lb{...}\rb} generates a box
containing {\tt...}, then places this box on the page in such a way that its
{\tt ij}-point ends up {\tt d truecm} down and {\tt r truecm} right of the
upper left corner of the grid generated by \tb{figinbox}. The altogether nine
{\tt ij}-points of a box are
\smallskip\centerline{\vbox{\tt\halign{#\hskip.5em&#\hskip.5em&#\cr
00&01&02\cr10&11&12\cr20&21&22\cr}}}
\smallskip To help in the placement, you can turn on a {\tt truecm} grid by
turning off that \tb{showgridfalse}.

\sect{Tables} Tables can be created with standard \Tex commands.
If you prefer, you can also create your table with La\TeX, convert
it to postscript, and then treat it like a figure. Table captions
should be created with the macro \tb{figcaption}.

\sect{Using Labels and Automatic Numbering} Some seem to
think that \Tex is inferior to La\Tex because only La\Tex can do
automatic numbering.  Nothing could be further from the truth.
Automatic numbering and labelling of equations and of formal statements
(Lemmas, Theorems, etc.), figures, tables, and references is easy
to do in \Tex using the numbering macros which are included in
macro file {\tt satmacros.tex}.

Whenever you want to number a formal statement, use
\tb{label}{\tt\lb<name>\rb} at the place at which the number is to appear.
(This is exactly how the numbering earlier, of Definition~\recall{def1}
and Theorem~\recall{thm2}, was managed in {\tt instruct.tex}.)
This causes \Tex to increment an internal count, \tb{labelnum}, by 1, place
that number there, and also remember the number under the name {\tt<name>}.

You can refer to this number anywhere in the text by
\tb{recall}{\tt\lb<name>\rb}. If you don't intend to refer to the number at
all, you can leave {\tt<name>} empty, i.e., use
\tb{label}{\tt\lb{}\rb}.

If you would like such numbering to restart in each section, make sure that
\tb{presect} is not redefined  in your \Tex file (compare the beginning of
{\tt instruct.tex} with that of {\tt textemplate.tex}).

The procedure is exactly the same if you would like to number an equation.
Where you would usually say \tb{eqno}{\tt(<number>)}, you would now say
\tb{label}{\tt\lb<name>\rb}. This causes \Tex to increment an internal count,
\tb{eqnum}, by 1, place it there inside parentheses, and remember the number
by the name {\tt<name>}.
 You would refer to this equation in the text by
the command
{\tt(}\tb{recall}{\tt\lb<name>\rb)}. Again, you could choose {\tt<name>} to
be empty if you don't intend to refer to this number in the text.

Special care must be taken when you want to give a number to equations that
are being aligned via \tb{eqalignno}. Use \tb{eqalignlabel} instead of
\tb{label} there.

Finally, you may wish to give an equation the same number as a previous
equation but distinguish the two by an additional letter, as in the
following variant of an earlier example:
$$\eqalignno{N& =  2n-n_b-2, \qquad 1 \le n \le 16, &\eqalignlabelplus{}a \cr
            e & =  3n-n_b-3, \qquad 1 \le n \le 4.&\samenumber\eqalignlabelplus{}b\cr}$$
Have a look at the file {\tt instruct.tex} to see how this is done using
\tb{samenumber} and \tb{eqalignlabelplus}. To be sure, there is also the macro
\tb{labelplus}, of use in equations that don't involve \tb{eqalignno}.

If you prefer to number all items by a single count (to make the search for a
particular item as easy as possible), be sure to turn on the \tb{singlecount}
macro at the beginning of your \Tex file.

Also, as you work on your paper, you may find that you forgot just what name
you have given a particular item. In that case, turn on the \tb{draft} macro
at the beginning of your \Tex file. This will cause all the names to be
printed in the margins next to their items.

Finally, if macro names like \tb{label} and \tb{recall} are too long for your
taste, you can also add to your macros at the beginning of your \Tex file

\leftline{\tb{let}\tb{en}\tb{label}}
\leftline{\tb{let}\tb{er}\tb{recall}}

\noindent and even

\leftline{\tb{def}\tb{eqr}{\tt\#1\lb(\tb{recall}\lb{\#1}\rb)\rb}}

\noindent and use these shorter versions, \tb{en}, \tb{er}, \tb{eqr}, in
your \Tex file.

\sect{Defining {\it vs\/} Emphasizing}
We have provided the macro \tb{dword} for you to mark any word or phrase when
it is first defined. This causes the word or phrase to be typeset in a
distinctive font and so helps the reader to see just what term in a formal
definition is being defined. For example, the file {\tt instruct.tex} uses
type {\tb{dword} \tt \lb{}triangulation\rb}
in Def.~\recall{def1} above.
\remember{\the\sectionno}{emph}

This also helps the reader locate the definition of a word or term,
particularly if that definition is not formal but happens as part of the text.
Systematic use of \tb{dword} also helps with the creation of an index.

We have also provided the macro \tb{eword} for you to emphasize any word or
phrase. At present, this causes the word or phrase to be typeset in italics.

\sect{A Few Other Macros}
The macros in {\tt satmacros.tex} are grouped according to function.
Besides the ones mentioned above, we have included a few others
which you may find useful.  For example, you may want to use
\tb{inpro}{\tt\lb{a,b}\rb} and \tb{norm}{\tt\lb{c}\rb} for the inner product
$\inpro{a,b}$ and the norm $\norm{c}$, respectively.
To get $\frac{a}{b}$,
you can type \$\tb{frac\lb{}a\rb\lb{}b\rb\$}.

We encourage you to use \tb{dd}, \tb{ee}, and \tb{ii} for the `d' in
integrals, the base for the natural logarithm, and the imaginary unit,
respectively, as in
$$\int_{-\pi}^\pi \ee^{\ii x}\dd x = 0.$$

\sect{Some Things Not to Do}
Here is a short list of some things you should {\bf not} do:
\remember{\the\sectionno}{nottodo}
\medskip
\bull Do not type your \Tex file using any special characters
you may have on your keyboard or in your editor.  This applies
especially to editors that have keys for letters with diacritical
marks like \"a, \"u, \'a, \`a, \~n, etc.
\smallskip
\bull Do not change anything in the macro file {\tt satmacros.tex}.
\smallskip
\bull Do not redefine any of the standard macros of \Tex
or any of the macros in {\tt satmacros.tex}.
Thus, before you define a macro for yourself, you should
check to make sure that it has not already been defined.
\smallskip
\bull Page size is not an issue. We want a uniform style and format. Do not
change \tb{hsize}, \tb{vsize}, \tb{baselineskip}, \tb{magstep},
or use different fonts.
\smallskip
\bull Do not add extra space between paragraphs.
\smallskip
\bull Do not add spaces before punctuation.  \Tex does a very
  good job of spacing, and we just have to edit all such extra
  space out.
\smallskip
\bull Do not use footnotes.  Simply put what you want to say in
  the text at the appropriate place (possibly in parentheses).
\smallskip
\bull Do not put a colon before every displayed equation.  The
  only correct time to use a colon there is if you are saying
  something like {\sl ... the following equation:}
\smallskip
\bull Do not simply copy references from some file you may have
  on hand.  We have a specific reference style which is spelled
  out in Sect.~\recall{refs} below. Please follow it.
\smallskip
\bull Do not use \tb{sl}, \tb{it}, or \tb{bf} to emphasize words.
Instead, use \tb{dword} and \tb{eword} as explained in
Sect.~\recall{emph}.
\smallskip
\bull When typing your \Tex file, break your sentences into
short pieces by hitting the \eword{return} or
\eword{enter} key. \Tex ignores blanks, and
it makes your file a lot easier to read and edit.

\sect{References} As in all good books and journals, we would like
to have a uniform way of listing references. This means different
styles for journal papers, proceedings papers, books, unpublished
reports or preprints, and dissertations. But, in order to save you the
trouble of learning the details, of various font choices, punctuation, and
the like, we have equipped {\tt satmacros.tex} with macros to handle
correctly any reference written in the minimal format employed in the Spline
Bibliography and explained in full detail in the file {\tt journal.tex}.
For examples, look at the file {\tt instruct.tex} which generated this
document.
\remember{\the\sectionno}{refs}

Having the references in this minimal format may also help you later on
since, by imitating how the macros in {\tt satmacrox.tex} convert these
references to SAT style, you can then convert the references to any style
required by any journal (see the file {\tt refmac.tex} for many examples).

Please prepare your reference list carefully according
to the following rules: \medskip

\bull  Arrange your references in alphabetical order.
\smallskip
\bull Preface each references with
one of \tb{refB}, \tb{refD}, \tb{refJ},  \tb{refQ}, or \tb{refR} depending on
whether the reference is a book, a dissertation, a journal article, a
proceedings paper, or an unpublished report or preprint.
\smallskip
\bull Then supply the author's name, in the form {\tt <lastname>, <firstname>;} .
If there is more than one author, repeat the pattern, i.e., supply the list
of authors in the form {\tt <lastname>, <firstname>, <lastname>, <firstname>,
... <lastname>, <firstname>;}, making sure there is a blank space after each
comma. Here, {\tt <firstname>} could be just the
initials, in which case, please, leave a space between them. E.g., type
{\tt Schoenberg, I. J.}, not {\tt Schoenberg, I.J.}.
\smallskip
\bull Then supply the title, also followed by a semicolon. If it is a book
or dissertation, capitalize all words but the inessential ones (just as
described earlier for section headings).
\smallskip
\bull Then supply the `locator' for the reference. This depends on the
particular kind of reference, as follows:
\bigskip
\rightline{\vbox{\halign{\hfil{\tt#}\hfil&\hskip.5em#\hfil&\hskip2em{\tt#}\cr
<style>&type&<locator>\cr
\noalign{\vskip4pt}
B&book&<publisher-info>; <year>;\cr
D&dissertation&<institution>; <year>;\cr
J&in journal&<journal>; <volume>; <year>; <pages>;\cr
Q&in proceedings&<proceedings>; <year>; <pages>;\cr
R&report&<report-no.\ and institution>; <year>;\cr}}}
\bigskip
\nobull
{\tt<publisher-info>} is to be of the form {\tt<publisher-name>
(<publisher-location>)}, and {\tt<publisher-name>} may not end in a forced
blank.
\smallskip
\nobull
{\tt<proceedings>} must be of the form {\tt(<title of proceedings>),}\
{\tt<editor name[s]> (ed[s].),}\ {\tt<publisher-info>},
 with each {\tt<editor name>} of the
form {\tt<editor initials> <editor last name>}. Note again the need for a
blank space after each comma in this form.
\smallskip
\nobull
{\tt<pages>} must have the page numbers separated by a double-dash; for example,
{\tt58--63}.
\smallskip
\nobull For {\tt<journal>}, you can save yourself some typing by using the
macros, such as \tb{JAT} or \tb{CA}, for standard journals available in the
file {\tt journal.tex}.
\smallskip
\nobull There is even more saving available for any reference in a standard
proceedings, as defined in the file {\tt proceed.tex}. For any such, you can
use the style \tb{refP\ }{\tt<author>; <title>; <proceedings-macro>;
<pages>;}.
\smallskip
\bull Notice that the minimal format consists of fields separated by
semicolons. If you need to use a semicolon as part of the reference itself,
enclose it in braces. Also, notice that each semicolon has to be either at
the end of a line or else be followed by at least one blank space.
\smallskip
\bull What if a reference does not at all fit one of these styles? Then
preface it with the command \tb{refX} and do the best you can (see, e.g.,
MacTutor [2004]).
\medskip
\noindent To cite references listed in the References section,
write the last name of the author followed by the year of publication in
brackets. If there is more than one author, follow the first author's name by
`et al.'.

%******************************* REFERENCES *************************

\References

\refQ  Boor, C. de;
The quasi-interpolant as a tool in elementary polynomial spline theory;
(Approximation Theory),
G. G. Lorentz {\it et al.} (eds.),
Academic Press (New York); 1973; 269--276;

\refD Hopf, Eberhard;
\"Uber die Zusammenh\"ange zwischen gewissen h\"oheren
   Diffe\-renzen\-quo\-tienten reeller Funktionen einer reellen Variablen und
  deren Dif\-fe\-renzier\-bar\-keits\-eigen\-schaften;
Universit\"at Berlin (30 pp); 1926;

\refR Lorentz,  G. G.;
Birkhoff interpolation problem;
CNA, U.Texas at Austin; 1975;

\refX MacTutor. [2004] {\tt http://www-groups.dcs.st-and.ac.uk/$\sim$history}

\refB Pinkus, A.;
$n$-Widths in Approximation Theory;
Springer-Verlag (New York); 1985;

\refP Schoenberg,  I. J.;
On variation diminishing approximation methods;
\Langer; 249--274;

\refJ Totik, V.;
Sharpness of Timan's converse result for polynomial approximation;
J.~Approx.~Theory; 67; 1991; 357--359;

\bye %% END of file instruct.tex
