 \documentclass[11pt]{article}
\usepackage{cv}
\name{Simon Foucart}
\info{Address: & Texas A\&M University\\
						& Department of Mathematics\\
                          %& 269 Korman Center, 3141 Chestnut Street\\
                          & College Station, TX 77843-3368\\ 
      %Phone: & +1 706-542-2570 \\
      Emails: & foucart@tamu.edu\\
     	          & simon.foucart@centraliens.net\\
      WWW: & \url{http://foucart.github.io}}

\begin{document}
\maketitle
 
\vspace{10mm}
%\section{Current and Past Positions}
%\begin{tabular}{ll}
%2018 & Visiting Researcher (Jun), LAAS-CNRS, Toulouse, France\\
%2017 & Visiting Researcher (Dec), Hong Kong University of Science and Technology, Hong Kong\\
%2015- & Associate Professor of Mathematics, Texas A\&M University, College Station\\
%2015 & Visiting Researcher (May-Jun), University of South Florida, Tampa\\
%2013-15 & Assistant Professor of Mathematics, University of Georgia, Athens\\
%2010-13 & Assistant Professor of Mathematics, Drexel University, Philadelphia\\
%2009-10 & Postdoctoral Researcher, Universit\'{e} Pierre et Marie Curie, Paris, France\\
%& (Laboratoire Jacques-Louis Lions; Mentor: Albert Cohen)\\
%2009 & Visiting Researcher (Jul-Aug), University of Bonn, Germany\\ 
%& (Hausdorff Center for Mathematics; Host: Holger Rauhut)\\
%2006-09 & Assistant Professor of Mathematics (NTT), 
%Vanderbilt University, Nashville\\ 
%& (Center for Constructive Approximation; Mentor: Larry Schumaker)
%\end{tabular}

%\section{Current and Past Positions}
\section{Primary Positions}
\begin{tabular}{ll}

2019- & Professor of Mathematics, Texas A\&M University, College Station\\
2015-19 & Associate Professor of Mathematics, Texas A\&M University, College Station\\
2013-15 & Assistant Professor of Mathematics, University of Georgia, Athens\\
2010-13 & Assistant Professor of Mathematics,  Drexel University, Philadelphia\\
2009-10 & Postdoctoral Researcher,  Universit\'{e} Pierre et Marie Curie, Paris, France\\
%& (Laboratoire Jacques-Louis Lions; Mentor: Albert Cohen)\\
2006-09 & Assistant Professor of Mathematics (NTT),  Vanderbilt University, Nashville
% & (Center for Constructive Approximation; Mentor: Larry Schumaker)
\end{tabular}


\section{Secondary Commitments} %Engagements}
\begin{tabular}{ll}
2021- \; \, & TAMIDS Associate Director for External Academic Engagement\\
& \quad Texas A\&M Institute of Data Science\\
2022- \; \, & CAMDA Director\\
& \quad Center for Approximation and Mathematical Data Analytics
\end{tabular}


\section{Visiting Positions}
\begin{tabular}{ll}
2023 & Visiting Researcher (Jul), Los Alamos National Laboratory\\
%2019 & Visiting Researcher (Jun-Jul), LAAS-CNRS, Toulouse, France\\
2019 & Visiting Researcher (Jan-May), Wisconsin Institute for Discovery, UW--Madison\\ 
2018 & Visiting Researcher (Jun), LAAS-CNRS, Toulouse, France\\
2017 & Visiting Researcher (Dec), Hong Kong University of Science and Technology, Hong Kong\\
2015 & Visiting Researcher (May-Jun), University of South Florida, Tampa\\
2009 & Visiting Researcher (Jul-Aug), University of Bonn, Germany
\end{tabular}
 

%\section{Higher Education}
%\section{Academic Training}
%\begin{tabular}{lll}
%2001-05 & PhD in Mathematics & University of Cambridge, U.K., Numerical Analysis Group\\
%& Advisor: Alexei Shadrin &\\
%2000-01 & Part III of Math Tripos & University of Cambridge, U.K.\\
%& With distinction & \\ 
%1998-01 & Masters of Engineering & Ecole Centrale Paris, France\\   
%1998-99 & Licence de Math\'{e}matiques & Universit\'{e} Pierre et Marie Curie, Paris, France\\
%\end{tabular}  


\section{Academic Training}
\begin{tabular}{lll}
2001-05 & PhD in Math (Numerical Analysis group) & University of Cambridge, U.K.\\
2000-01 & Part III of Math Tripos (with distinction) & University of Cambridge, U.K.\\
1998-01 & Masters of Engineering & Ecole Centrale Paris, France\\   
1998-99 & Licence de Math\'{e}matiques & Universit\'{e} Pierre et Marie Curie,  Paris, France\\
\end{tabular}  

\section{Research Interests}

Mathematical Data Science;
Compressive Sensing;
Approximation Theory; 
Computational Mathematics;
Bioinformatics 

%Compressive Sensing;
%Approximation Theory (especially Spline Functions and Minimal Projections); 
%Data Science;
%Computational Mathematics;
%Bioinformatics 
%Abstract and Classical Analysis

\section{Honors and Awards}

\begin{tabular}{ll}
2019 & Recipient of a {\sl Presidential Impact Fellowship}, Texas A\&M University\\
2012 & Recipient of the  {\sl Antelo Devereux Award for Young Faculty}, Drexel University\\
2010 & {\sl Journal of Complexity} Best Paper Award\\
%`The Gelfand widths of $\ell_p$-balls for $0<p \le 1$' with A. Pajor, H. Rauhut, T. Ullrich.
2000-04 & Various scholarships received at the University of Cambridge\\
& (Dept  of Applied Math and Theoretical Physics; Trinity Hall; Cambridge European Trust)\\
2001 & Scholar of Trinity Hall, %University of Cambridge, 
added to the College Register
\end{tabular}
%Travel Grants, University of Cambridge, 2003 and 2005.\\
% Scholarships received at the University of Cambridge
% from the Department  of Applied Mathematics and Theoretical
% Physics (2001-2004),
% from Trinity Hall (2001-2002), and 
%from the Cambridge European Trust 
% (2000-2001 and 2001-2004)

 
%\section{Research Grants}
\section{External Funding}

%`Improving Analysis of Microbial Mixtures through Sparse
%Reconstruction and Statistical Inference'\\
%\hspace{18mm}PI; coPIs: Gail Rosen (Drexel Engineering) and Loni Philip Tabb (Drexel Biostatistics)\\
%\hspace{18mm}NSF grant DMS-1120622, \$666,322, Sep 2011-Aug 2014 (extended until Aug 2015)

\begin{tabular}{lll}
%2023-28 & NSF; senior personal,  PI: S.  Banerjee (TAMU Chemistry); \$30,000,000\\
%& \hspace{-15mm} 
%{\small STC:} Center for Mathematical, Molecular, and Materials Foundations of Complementary Intelligence\\
2023 & NSF; co-PI,  PI: S. Wojtowytsch (TAMU Math); \$35,200\\
& \hspace{-15mm}
{\small Conference:} Inaugural CAMDA Conference\\
2021-24 & NSF; sole PI; \$149,783\\
& \hspace{-15mm}
{\small CDS\&E-MSS:} Optimal Recovery in the age of Data Science\\
2020-23 & ONR; local CoPI, local PI: R.~DeVore (TAMU Math); \$883,622;
Lead: Rice University\\
& \hspace{-15mm}
{\small MURI:} Theoretical foundations of Deep Learning \\
2019-22 & NSF; senior personnel (executive committee), PI: B.~Mallick (TAMU  Statistics);  \$1,416,522 \\
& \hspace{-15mm}
{\small TRIPODS:} Texas A\&M Research Institute for Foundations of Interdisciplinary Data Science\\
2018-21 & NSF; coPI, PI: D.~Koslicki (Oregon State Math), coPI: I.~Ivanov (TAMU Vet Med); 
\$292,041 \\
& \hspace{-15mm}
{\small QuBBD:} Fast, efficient mathematical approach to the analysis of the
human microbiome through\\ 
& \hspace{-15mm} biodiversity optimization\\
2016-19 & NSF; sole PI;
\$99,535 
%;\$257,130
\\
& \hspace{-15mm}
{\small CDS\&E-MSS:} Recovery of high-dimensional structured functions
\\
2011-15 & NSF; PI, coPIs: G. Rosen (Drexel Engineering), L. P. Tabb (Drexel Biostatistics); 
\$666,322\\
%grant DMS-1120622\\
& \hspace{-15mm}
{\small ATD:} Improving analysis of microbial mixtures through sparse
reconstruction and statistical inference
\end{tabular}

\section{Internal Funding}

\begin{tabular}{lll}
2023 & Texas A\&M ASCEND Initiative; coPI, PI: R. Arroyave (Engineering), \$464K\\
& \hspace{-15mm}
%{\small TPT:} Autonomous Scientific Discovery
{\small TPT:} Autonomous Materials Discovery\\
2023 & Texas A\&M University System National Laboratories Office, \$32K\\
& \hspace{-15mm}
Development fellowship to strengthen collaborations with Los Alamos National Laboratory\\
2022 & College of Arts and Sciences;
coPI, PI: T. Logan (Atmospheric Sciences),  \$10K\\
& \hspace{-15mm}
{\small Seed Grant Promoting Research Collaborations:}
Do HLMA data contain evidence of space lightning?\\
2022 & Texas A\&M University System National Laboratories Office, \$1.5K\\
& \hspace{-15mm}
Exploration minigrant to initiate collaborations with Los Alamos National Laboratory\\
2021 & Texas A\&M Institute of Data Science, \$15K\\
& \hspace{-15mm}
TAMIDS Course Development Grant for {\em MATH 664: Topics in Mathematical Data Science}\\
2021-22 & Texas A\&M; PI,  CoPI: R.~Tuo (Engineering), S.~Shahrampour (now at Northeastern), \$30K\\
& \hspace{-15mm}
{\small T3 Triads:} Learning more efficiently with less labels\\
2019-20 & Texas A\&M; coPI, PI: S.~Shahrampour (Engineering), CoPI: B.~Hanin (Math), \$32,876\\
& \hspace{-15mm}
{\small T3 Triads:} Trade-offs between approximation and generalization in learning systems
\end{tabular}


%\section{Honors and Awards}
%
%\begin{tabular}{ll}
%2019 & Recipient of a {\sl Presidential Impact Fellowship}, Texas A\&M University\\
%2012 & Recipient of the  {\sl Antelo Devereux Award for Young Faculty}, Drexel University\\
%2010 & {\sl Journal of Complexity} Best Paper Award\\
%%`The Gelfand widths of $\ell_p$-balls for $0<p \le 1$' with A. Pajor, H. Rauhut, T. Ullrich.
%2000-04 & Various scholarships received at the University of Cambridge\\
%& (Dept  of Applied Math and Theoretical Physics; Trinity Hall; Cambridge European Trust)\\
%2001 & Scholar of Trinity Hall, %University of Cambridge, 
%added to the College Register
%\end{tabular}
%%Travel Grants, University of Cambridge, 2003 and 2005.\\
%% Scholarships received at the University of Cambridge
%% from the Department  of Applied Mathematics and Theoretical
%% Physics (2001-2004),
%% from Trinity Hall (2001-2002), and 
%%from the Cambridge European Trust 
%% (2000-2001 and 2001-2004)


\section{Publications}

\subsection{Books}
\betaremune
\item{\sl Mathematical Pictures at a Data Science Exhibition.}\\
Cambridge University Press, 2022. 
\item {\sl A Mathematical Introduction to Compressive Sensing.}\\
Birkh\"auser, Applied and Numerical Harmonic Analysis, 2013.  With H. Rauhut.
\eetaremune

\subsection{Surveys}
\betaremune
\item {\sl Flavors of Compressive Sensing}.\\
Approximation Theory XV: San Antonio 2016,
Springer Proceedings in Mathematics \& Statistics,
vol~201, 61--104. 
\eetaremune

\subsection{Refereed Journal Papers}
\betaremune
\item {\sl Near-optimal estimation of linear functionals with log-concave observation errors.}\\
Information and Inference, 12/4, iaad038, 2023.  
With G. Paouris.
\item {\sl Full recovery from point values: an optimal algorithm for Chebyshev approximability prior.}\\
Advances in Computational Mathematics,  49, 57, 2023.	
\item {\sl The sparsity of LASSO-type minimizers.}\\
Applied and Computational Harmonic Analysis, 62, 441--452, 2023.
% to Mathematical Statistics and Learning
\item {\sl On the value of the fifth maximal projection constant.}\\
Journal of Functional Analysis, 283/10, 109634, 2022.
With B. Deregowska, M. Fickus,  B. Lewandowska.
\item {\sl On the sparsity of LASSO minimizers in sparse data recovery.}\\
Constructive Approximation, 57, 901--919, 2023. With E. Tadmor, M.  Zhong.
\item {\sl Optimal recovery from inaccurate data in Hilbert spaces: regularize, but what of the parameter?}\\
Constructive Approximation, 57, 489--520, 2023.  With C.~Liao.
\item {\sl Learning from non-random data in Hilbert spaces: an optimal recovery perspective.}\\
Sampling Theory, Signal Processing, and Data Analysis,
20, 5, 2022.
With C.~Liao, S.~Shahrampour, Y.~Wang. 
\item {\sl Instances of computational optimal recovery: dealing with observation errors.}\\
SIAM/ASA Journal on Uncertainty Quantification, 9/4, 1438--1456, 2021. 
With M.~Ettehad.
\item {\sl Raconte-moi ... le Compressive Sensing.}\\
La Gazette des Math\'ematiciens, 168, 2021. 
\item {\sl Weighted matrix completion from non-random, non-uniform sampling patterns.}\\
IEEE Transactions on Information Theory, 67/2, 1264--1290, 2021.
With D.~Needell, R.~Pathak, Y.~Plan, \mbox{M.~Wootters}.
\item {\sl Nonlinear approximation and (deep) ReLU networks.}\\
Constructive Approximation,  55, 127--172, 2022.
With I.~Daubechies, R.~DeVore, B.~Hanin, G.~Petrova.
\item {\sl Instances of computational optimal recovery: refined approximability models.}\\
Journal of Complexity, 62, 101503, 2021.
\item {\sl Facilitating OWL norm minimizations.}\\
Optimization Letters, 15/1, 263--269, 2021.
\item {\sl Approximability models and optimal system identification.}\\
%SIAM Journal on Control and Optimization.
Mathematics of Control, Signals, and Systems, 32/1, 19--41, 2020.
With M.~Ettehad.
\item {\sl Sampling schemes and recovery algorithms for functions of few coordinate variables.}\\
Journal of Complexity, 58, 101457, 2020.
\item {\sl Jointly low-rank and bisparse recovery: questions and partial answers.}\\
Analysis and Applications (special issue on Mathematics of Data Science), 18/1, 25--48, 2020.
With R.~Gribonval, L.~Jacques, H.~Rauhut.
\item {\sl Computation of Chebyshev polynomials for union of intervals.}\\
Computational Methods and Function Theory, 19/4, 625--641, 2019.
With J. B. Lasserre.
\item {\sl Optimal algorithms for computing average temperatures.}\\
%Journal of Atmospheric and Oceanic Technology.
Mathematics of Climate and Weather Forecasting, 5, 34--44, 2019.
With M. Hielsberg, G.~Mullendore, G.~Petrova, P. Wojtaszczyk.
\item {\sl Iterative hard thresholding for low-rank recovery from rank-one projections}.\\
Linear Algebra and its Applications, 572, 117--134, 2019. With S. Subramanian.
\item {\sl Recovering low-rank matrices from binary measurements}.\\
Inverse Problems and Imaging, 13/4, 703--720, 2019. 
With R. Lynch. 
\item {\sl Determining projection constants of univariate polynomial spaces}.\\
Journal of Approximation Theory, 235, 74--91, 2018.
With J. B. Lasserre.
\item {\sl Computing a quantity of interest from observational data}.\\
Constructive Approximation, 49/3, 461--508, 2019. With R. DeVore, G. Petrova, P. Wojtaszczyk.
\item {\sl Sparse recovery from inaccurate saturated measurements}.\\
Acta Applicandae Mathematicae, 158/1, 49--66, 2018. With J. Li.
\item {\sl On the norms and minimal properties of de la Vall\'ee Poussin's type operators}.\\
Monatshefte f\"ur Mathematik, 185/4, 601--619, 2018. With B.~Deregowska, B.~Lewandowska, L.~Skrzypek.
\item {\sl Concave Mirsky inequality and low-rank recovery.}\\
SIAM Journal on Matrix Analysis and Applications, 39/1, 99--103, 2018.
\item {\sl An IHT algorithm for sparse recovery from subexponential measurements}.\\
IEEE Signal Processing Letters, 24/9, 1280--1283, 2017. 
With G.~Lecu\'e.
\item {\sl One-bit compressive sensing of dictionary-sparse signals.}\\
Information and Inference, 7/1, 83--104, 2018.
With R. Baraniuk, D. Needell, Y. Plan, M.~Wootters.
\item {\sl Exponential decay of reconstruction error from binary measurements of sparse signals.}\\
IEEE Transactions on Information Theory, 63/6, 3368--3385, 2017.
With R. Baraniuk, D. Needell, Y.~Plan, and \mbox{M.~Wootters}.
\item {\sl On maximal relative projection constants.}\\
Journal of Mathematical Analysis and Applications,  447/1, 309--328, 2017.
With L. Skrzypek.
\item {\sl  Sparse recovery from saturated measurements.}\\
Information and Inference, 6/2, 196--212, 2017.
With T. Needham.
\item{\sl {\sf Basc}: constrained approximation by semidefinite programming.}\\
IMA Journal of Numerical Analysis, 37/2, 1066--1085, 2017.
With V. Powers.
\item {\sl Hard thresholding pursuit algorithms: number of iterations.}\\
Applied and Computational Harmonic Analysis, 
41/2, 412--435, 2016.
With J.-L. Bouchot, P. Hitczenko. 
\item {\sl Computation of minimal projections and extensions.}\\
Numerical Functional Analysis and Optimization. 37/2, 159--185, 2016.
\item {\sl Dictionary-sparse recovery via thresholding-based algorithms.}\\ 
Journal of Fourier Analysis and Applications. 22/1, 6--19, 2016.
\item {\sl Sparse disjointed recovery from noninflating measurements.}\\
Applied and Computational Harmonic Analysis, 39/3, 558--567, 2015.
With M. Minner, T. Needham.
\item {\sl WSGQuikr: fast whole-genome shotgun metagenomic classification.}\\
PLoS ONE, 9/3, e91784, 2014. 
With D. Koslicki, G. Rosen.
\item{\sl Sparse recovery by means of nonnegative least squares.}\\
IEEE Signal Processing Letters, 21/4, 498--502, 2014. 
With D. Koslicki.
\item {\sl Quikr: a method for rapid reconstruction of bacterial communities via compressive sensing.}\\
Bioinformatics, 29/17, 2096--2102, 2013. 
With D. Koslicki, G. Rosen.
\item {\sl Generating dimension formulas for multivariate splines.}\\
Albanian Journal of Mathematics, 7/1, 24--35, 2013.
 With T. Sorokina.
\item {\sl Stability and robustness of $\ell_1$-minimizations with Weibull matrices and redundant dictionaries.}\\
Linear Algebra and its Applications, 441, 4--21, 2014.
\item {\sl Hard thresholding pursuit: an algorithm for Compressive Sensing.}\\
SIAM Journal on Numerical Analysis, 49/6, 2543--2563, 2011.
\item {\sl The Gelfand widths of $\ell_p$-balls for $0<p \le 1$.}\\
Journal of Complexity, 26/6, 629--640, 2010. 
With A. Pajor, H. Rauhut, T. Ullrich.
\item {\sl Real versus complex null space properties for sparse vector recovery.}\\
Comptes Rendus de l'Acad\'emie des Sciences, 348, 863--865, 2010. 
With R. Gribonval.
\item {\sl A note on guaranteed sparse recovery via $\ell_1$-minimization.}\\
Applied and Computational Harmonic Analysis, 29/1, 97--103, 2010.
\item {\sl Sparse recovery with pre-Gaussian random matrices.}\\
Studia Mathematica, 200, 91--102, 2010. 
With M.-J. Lai.
\item {\sl Allometry constants of finite-dimensional spaces: theory and computations.}\\
Numerische Mathematik,  112/4, 535--564, 2009.
\item {\sl  Sparsest solutions of underdetermined linear systems via $\ell_q$-minimization for $0 <  q  \le  1$.}\\
Applied and Computational Harmonic Analysis, 26/3, 395--407, 2009. 
With M.-J. Lai.
\item {\sl Open questions around the spline orthoprojector.}\\
East Journal on Approximations, 14/2, 241--253, 2008.
\item {\sl On the exact constant in Jackson--Stechkin inequality for the uniform metric.}\\
Constructive Approximation, 29/2, 157--179, 2009. 
With Yu. Kryakin, A. Shadrin.
\item {\sl On the value of the max-norm of the orthogonal projector onto splines with multiple knots.}\\
Journal of Approximation Theory, 140/2, 154--177, 2006.
\item {\sl  Interlacing property for B-splines.}\\
Journal of Approximation Theory, 135/1, 1--21, 2005.
\item {\sl On the best conditioned bases of quadratic polynomials.}\\
Journal of Approximation Theory, 130/1, 46--56, 2004.
\eetaremune
	

\subsection{Refereed Proceedings Papers}
\betaremune
\item {\sl On the optimal recovery of graph signals.}\\
SampTA 2023, New Haven.  With C. Liao, N. Veldt.
% to SampTA
\item {\sl Finer metagenomic reconstruction via biodiversity optimization.}\\
NeurIPS 2020, Vancouver (online).
%to Journal of Mathematical Biology. 
With D.~Koslicki.
\item {\sl One-bit sensing of low-rank and bisparse matrices}.\\
%Proceedings of 
SampTA 2019, Bordeaux. With L.~Jacques.
\item {\sl De-biasing low-rank projection for matrix completion.}\\
%Proceedings of  
SPIE Optics and Photonics, San Diego 2017.
With D. Needell, Y. Plan, M. Wootters.
\item {\sl Complexity of multivariate problems based on binary information}.\\
%Proceedings of 
SampTA 2017, Tallinn.
\item {\sl Stability and robustness of weak orthogonal matching pursuits.}\\
%To appear in:
In: Recent Advances in Harmonic Analysis and  Applications,
% {\sl AMS Spring 2011 Southeastern Conference},
Springer Proceedings in Mathematics \& Statistics, vol 25, 395--405.
\item {\sl Recovering jointly sparse vectors via hard thresholding pursuit.}\\ 
%Proceedings of 
SampTA 2011, Singapore.
\item {\sl Recovery of functions of many variables via compressive sensing.}\\
%Proceedings of 
SampTA 2011, Singapore. 
With A. Cohen, R. DeVore, H. Rauhut.
\item {\sl Sparse recovery algorithms: sufficient conditions in terms of restricted isometry constants.}\\
In: Approximation Theory XIII: San Antonio 2010, Springer Proceedings in Mathematics, vol 13, 65--77.
\item {\sl  Some comments on the comparison between condition numbers and projection constants.}\\
In: Approximation Theory XII: San Antonio 2007, Nashboro Press, 143--156. 
%M. Neamtu and  L. L. Schumaker (eds.), 
\eetaremune

\subsection{Working Papers}
\betaremune
%\item Multifidelity
\item {\sl Optimization-aided construction of multivariate Chebyshev polynomials}.\\
In preparation. With M.  Dressler, E. de Klerk, M.  Joldes, J.~B. Lasserre,  Y. Xu.
\item {\sl Radius of information for two intersected centered hyperellipsoids and implications in optimal recovery from inaccurate data}. \\
In preparation. With C. Liao.
\item {\sl Linearly embedding sparse vectors from $\ell_2$ to $\ell_1$
via deterministic dimension-reducing maps.}\\
Submitted.
\item {\sl S-procedure relaxation: a case of exactness involving Chebyshev centers}.\\
Submitted. With C. Liao.
\eetaremune

\subsection{Not for Publication}
\betaremune
\item {\sl Three topics in multivariate spline theory.}
\item {\sl Symbolic spline computations.}\\
 With P. Clarke.
\item {\sl On the Hermite spline conjecture and its connection to $k$-monotone densities.}\\
With F. Balabdaoui, J. Wellner.
\eetaremune

\subsection{Theses}
\begin{tabular}{ll}
PhD Dissertation & {\sl Small-normed projections onto polynomial and spline spaces.}\\
Part III Essay & {\sl On definitions of discrete topological chaos and their relations on intervals.}
\end{tabular}


\section{Oral Presentations}

\subsection{Popular Talks}
\betaremune
\item {\sl Compressive Sensing: Making the most of few measurements.}
Drexel University, Dean's seminar, 20~Apr 2011.
\eetaremune

\subsection{Plenary Addresses}
\bitemize
\item {\sl TBA}, Function Spaces XIII, Poznan, Poland,  8-13 Jul 2024.
{\small Postponed  from 2021 due to coronavirus.} %from 5-9 Jul 2021. 
\item {\sl Integrating Observation Errors in Optimal Recovery}, Focus program `Data Science, Approximation Theory, and Harmonic Analysis', Fields Institute, Toronto,  9 May-10 Jun 2022.
{\small (Postponed from 2021 due to coronavirus.} %17~May-11 June 2021).
\item {\sl Standard, One-Bit, and Saturated Compressive Sensing}, 4th international Traveling Workshop on Interactions between low-complexity data models and Sensing Techniques (iTWIST),
Marseille, France, 21-23 Nov 2018.
\item {\sl Assimilating Data to Optimally Compute Quantities of Interest}, 7th International Conference on Computational Harmonic Analysis,
Nashville, 14-18 May 2018.
\item {\sl Flavors of Compressive Sensing}, 
15th International Conference on Approximation Theory, San Antonio, 22-26 May 2016.
\eitemize

\subsection{Colloquia}
\bitemize
\item {\sl TBA},  
%CMS Keller Colloquium
California Institute of Technology,  8 April 2024.
\item {\sl A Trustworthy Learning Theory? \hspace{-1mm}The View from Optimal Recovery}, University at Albany, 29~Mar~2023.
\item {\sl Singular Flavors of Compressive Sensing}, Colorado State University, 10 Oct 2022.
%\item {\sl TBA}, University of Houston, 7 Feb 2020.
\item {\sl Optimal Recovery under Approximability Models, with Applications}, Michigan State University, 3~Dec~2018.
\item {\sl Standard, One-Bit, and Saturated Compressive Sensing},
University of Houston, 12 Sep 2018.
\item {\sl Excursion into the Mathematics of Compressive Sensing},
Texas A\&{M} University, 30 Jan 2015.
\item {\sl Sparse Recovery: an Overview Leading to $\ell_1$-Minimizations from Weibull Measurements},
University of Georgia, 10 Dec 2012.
\item {\sl Compressive Sensing and Banach Space Geometry},
Drexel University, 26 May 2011.
\item {\sl Compressive Sensing and the Hard Thresholding Pursuit algorithm},
 Towson University, 22 Apr 2011.
\item {\sl Recovery Algorithms in Compressive Sensing}, University of South Florida, 10 Dec 2010.
\item {\sl Compressive Sensing: the Optimization Approach}, Drexel University, 23 Apr 2009. 
\item {\sl From Approximation Theory to Compressive Sampling via Banach Space Geometry---a Computational Tour},
University of Georgia, 5 Feb 2008,  University of South Florida, 15  Feb 2008.
\eitemize

\subsection{Short Courses}
\bitemize
\item {\em Flavors of Compressive Sensing}, Doctoral School of the 4th international Traveling Workshop on Interactions between low-complexity data models and Sensing Techniques (iTWIST),
Marseille, France, 19-20 Nov 2018
\item {\em The Fundamentals of Compressive Sensing}, as part of the
HKUST--ICERM Visiting Fellow Program,
Hong Kong University of Science and Technology, 6-22 Dec 2017.
\item {\em Essentials of Compressive Sensing}, Winter School at the Trimester Program on `Mathematics of Signal Processing', Hausdorff Research Institute,
Bonn, Germany, 11-15 Jan 2016.
\item {\em A Mathematical Overview of Compressive Sensing},
University of South Florida, 18-22 May 2015.
\item {\em A Tutorial on Compressive Sensing},
CIMPA school on `New Trends in Applied Harmonic Analysis:
Sparse Representations, Compressed Sensing, and Multifractal Analysis',
Mar del Plata, Argentina, 5-16  Aug 2013.
\item {\em Les Math\'ematiques du Compressive Sensing --- une Introduction},
Labotatoire Paul Painlev\'e, Universit\'e des Sciences et Technologies de Lille, France, 20-22 Mar 2013. 
\eitemize

\subsection{Invited Workshop and Conference Presentations}
\bitemize 
\item {\sl TBA},
Workshop `Recent Progress on Optimal Point Distributions and Related Fields',
ICERM, Providence,  3-7 Jun 2024.
\item {\sl Three vignettes in computational optimal recovery}, 
Workshop `Computational Harmonic Analysis and Data Science',
Foundations of Computational Mathematics conference, 
Paris, France, 12-21 Jun 2023.
\item {\sl  On the optimal recovery of graph signals},
Minisymposium `Approximation Theory in Data Analysis and Deep Learning',
% previously `Time evolving signals, sampling, and learning',
International Conference on Approximation Theory and Beyond,
Nashville, 15-18 May 2023. {\small Postponed from 2020 due to coronavirus.}
\item {\sl Connections between minimal projections and equiangular tight frames}, Special session `Harmonic Analysis and its Applications to Signals and Information', AMS Central Meeting,  Cincinnati,  15-16 Apr 2023.
\item {\sl Recovery from corrupted data: recent results for various models},  Special session `Mathematics of Information',
Pacific Rim Mathematical Association (PRIMA) Congress, 
Vancouver, Canada,  4-9 Dec~2022.
\item {\sl On LASSO-type regularizations and sparsity of their minimizers}, Minisymposium `Greedy and sparse approximation',
10th International Conference on Curves and Surfaces,
Arcachon, France, 20-24 Jun 2022.
\item {\sl Optimal recovery from inaccurate data in Hilbert spaces},
Workshop `Mathematics of Data Science',
Hausdorff Research Institute, Bonn, Germany,  25-29 Apr 2022.
%({\small postponed from 27~Apr-1~May 2020}).
\item {\sl Integrating observation errors in optimal recovery},
TAMIDS workshop `Uncertainty Quantification: Theory Meets Practice',
College Station, 5 Nov 2021.
\item {\sl Nonlinear approximation and (deep) ReLU networks}, Minisymposium `Approximation theory of neural networks',
SIAM Annual Meeting, Spokane (online),
19-23 Jul 2021.
\item {\sl Restricted isometry properties and their role in compressive sensing},
Online workshop `High-dimensional covariance matrices, networks and concentration inequalities', 20-24 May 2021.
\item {\sl TBA},
`Approximation and Geometry in High Dimensions' conference,
Mathematical Research and Conference Center, Bedlewo, Poland,
16-22 Aug 2020.
{\small Postponed due to coronavirus.}
\item {\sl Recovering low-rank matrices from binary measurements}, 
Special session
`Applications of Computational and Compressive Imaging',
SIAM Conference on Imaging Science / SIAM Annual Meeting,
Toronto, Canada, 6-10 Jul 2020.
{\small Canceled due to coronavirus.}
\item {\sl TBA},
Workshop `Computational Harmonic Analysis and Compressive Sensing',
Foundations of Computational Mathematics conference, 
Vancouver, Canada, 15-24 Jun 2020. {\small Canceled due to coronavirus.}
%\item {\sl TBA},
%Minisymposium `Time evolving signals, sampling, and learning',
%International Conference on Approximation Theory and Beyond,
%Nashville, 11-14 May 2020. {\small Postponed due to coronavirus.}
%{\small Postponed due to coronavirus.}
\item {\sl Nonlinear approximation and (deep) ReLU networks},
Special session `Mathematical Analysis in Data Science',
Joint Mathematics Meetings,
Denver, 15-18 Jan 2020.
\item {\sl Functions of few coordinate variables: sampling schemes and recovery algorithms},
Minisymposium  `Recent Advances in High-Dimensional Approximation',
2nd Annual Meeting of SIAM Texas-Louisiana,
1-3 Nov 2019.
\item {\sl Sparse recovery techniques in metagenomics}, Workshop `Nonlinear Approximation',
University of South Carolina, Columbia, 25-27~Oct~2019.
\item {\sl One-bit sensing of low-rank and bisparse matrices}, 
Special session `Mathematical Theory of Quantization',
13th International Conference on Sampling Theory and Applications, 
Bordeaux, France,  \mbox{8-12~Jul~2019}.
\item {\sl Nonlinear approximation and (deep) ReLU networks},
3rd International Conference on Mathematics of Data Science,
Hong Kong, 19-23 Jun 2019.
\item {\sl Functions of few coordinate variables: sampling schemes and recovery algorithms},
 Workshop  `Approximation, Sampling, and Compression in High Dimensional Problems',
Isaac Newton Institute, Cambridge, U.K., 17-21 Jun 2019. 
\item {\sl Approximability models and optimal system identification}, Minisymposium `Theory and Algorithms for Improved Performance of Machine Learning in Scientific Applications',
SIAM Conference on Computational Science and Engineering, Spokane, 25~Feb-1~Mar 2019.
\item {\sl Assimilating data to optimally compute quantities of interest}, Minisymposium `Sparsity-Based \mbox{Methods} for High-Dimensional Approximation in Uncertainty Quantification',
International Conference on Spectral and High Order Methods,
London, U.K., 9-13 Jul~2018.
\item {\sl Semidefinite programming in approximation theory: two examples}, Workshop `Numerical Analysis and Approximation Theory meet Data Science', Banff, Canada, 22-27 Apr 2018.
\item {\sl Assimilating data to optimally compute quantities of interest}, 
Texas A\&M workshop `Big Data -- Data Driven Discovery', 
College Station, 20 Apr 2018.
\item {\sl The usefulness of a modified restricted isometry property}, `February Fourier Talks', University of Maryland, 15-16 Feb 2018.
\item {\sl Computing a quantity of interest from observational data}, 
Special session `Compressed Sensing and Machine Learning', 
Data Institute Conference, San Francisco, 15-17 Oct 2017. 
\item {\sl Concave Mirsky inequality and low-rank recovery}, Minisymposium `Compressed Sensing and Matrix Completion',
21st Meeting of the International Linear Algebra Society, Ames, 24-28 Jul 2017.
\item {\sl On maximal relative projection constants}, Summer Informal Regional Functional Analysis Seminar,
College Station, 21-23 Jul 2017.
\item {\sl Computing a quantity of interest from observational data} and {\sl The usefulness of a modified restricted isometry property}, 
Workshops on `Approximation Theory' and on `Computational Harmonic Analysis and Compressive Sensing',
Foundations of Computational Mathematics conference, 
Barcelona, Spain, 10-19 Jul 2017.
\item {\sl Complexity of multivariate problems based on binary information},
Special session `Mathematical Theory of Quantization',
12th International Conference on Sampling Theory and Applications, 
Tallinn, Estonia, 3-7 Jul 2017.
\item {\sl Computing a quantity of interest from observational data},
Workshop `Data-Driven Model Reduction',
College Station, 27 Apr 2017.
\item {\sl Computing a quantity of interest from observational data},
Workshop `Multiscale and High-Dimensional Problems', Oberwolfach, Germany, 
26 Mar-1~Apr 2017.
\item {\sl Computing a quantity of interest from observational data}, 1st International Conference on Mathematics of Data Science,
Hong Kong, 20-24 Mar 2017.
%\item {\sl TBA}, International Conference on Some Mathematical Approximation Approaches in Data Science, Hangzhou, China, 12-15 Dec 2016.
\item {\sl Sparse recovery via nonconvex optimization, with application in metagenomics},
Special session `Nonconvex and Non-Lipschitz Optimization',
5th International Conference on Continuous Optimization, Tokyo, Japan, 6-11 August 2016.
\item{\sl One-bit compressive sensing of dictionary-sparse signals},
Minisymposium `Compressive Sensing: Approximation and Optimization',
15th International Conference on Approximation Theory, San Antonio, 22-26 May 2016.
\item {\sl Sparse recovery from saturated measurements},
Workshop on `Challenges in High-Dimensional Analysis and Computation',
San Servolo, Italy, 1-5 May 2016.
\item {\sl How {\sc matlab} impacts my research},
Workshop `Scientific Computing with {\sc matlab} at Texas A\&M',
College Station, 25 Apr 2016. 
\item {\sl Sparse recovery from saturated measurements},
Special session `Trends in the Mathematics of Signal Processing and Imaging',
Joint Mathematical Meetings, Seattle, 6-9 Jan 2016.
\item {\sl Exponentially decaying error rate in one-bit compressive sensing}, `Information-based Complexity' conference, 
Mathematical Research and Conference Center, Bedlewo, Poland,
26 Apr-2 May 2015.
\item {\sl Dimensions of spline spaces, Dehn--Sommerville equations, and Schumaker’s conjecture},
Workshop on `Multivariate Splines and Algebraic Geometry', Oberwolfach, Germany, 19-25 Apr 2015.
\item {\sl Semidefinite programming for constrained approximation}, Special session `Approximation Theory in Signal Processing and Computer Science', AMS \mbox{Central} Meeting, East Lansing, 13-15 Mar 2015.
\item {\sl Recovery of signals with sparse frame expansions}, Special session `Frames and their Applications', Joint Mathematical Meetings, San Antonio, 10-13 Jan 2015. 
\item {\sl Using semidefinite programming in Approximation Theory}, Workshop `Approximation Theory',
Foundations of Computational Mathematics conference, Montevideo, 11-20 Dec 2014.
\item {\sl Exponentially decaying error rate in one-bit compressive sensing}, Workshop `Approximation, Integration, and Optimization',
ICERM, Providence, 29 Sep-3 Oct 2014. 
\item {\sl Exponentially decaying error rate in one-bit compressive sensing}, Minisymposium `Mathematics of Information and Low Dimensional Models',
SIAM Annual Meeting, Chicago, 7-11 Jul 2014.
\item {\sl Exponentially decaying reconstruction error in one-bit compressive sensing},  
5th International Conference on Computational Harmonic Analysis, Nashville, 19-23 May 2014.
\item {\sl New iterative algorithms in sparse approximation}, Special session `Approximation Theory in Signal Processing', AMS Central Sectional Meeting, Lubbock, 11-13 Apr 2014.
\item {\sl A snapshot of iterative algorithms for sparse recovery}, Georgia Scientific Computing Symposium, Kennesaw State University, 22 Feb 2014.
\item {\sl Computing dimension formulas for multivariate spline spaces}.
Minisymposium `Multivariate Splines',
14th International Conference on Approximation Theory, San Antonio, 7-10 Apr 2013.
\item  {\sl Stability and robustness of weak orthogonal matching pursuits.}
Special session `Models and \mbox{Applications} in Compressive Imaging', SIAM conference on Imaging Science, Philadelphia, 20-22 May 2012.
\item {\sl Stability and robustness of $\ell_1$-minimizations with Weibull matrices and redundant dictionaries.}
Workshop on `Probabilistic Techniques and Algorithms', University of Texas, 6-8 Apr 2012.
\item {\sl Hard Thresholding Pursuit: an algorithm for Compressive Sensing} and {\sl The dimension of trivariate spline spaces on Alfeld splits.}
Special sessions `Compressed Sensing' and `Multivariate Splines', International Symposium in Approximation Theory, Nashville, 17-21 May 2011.
\item {\sl Recovering jointly sparse vectors via Hard Thresholding Pursuit.} 
Special session `Sparse Approximation', 9th International Conference on Sampling Theory and Applications, Singapore, 2-6 May 2011.
\item {\sl Hard Thresholding Pursuit for sparse reconstruction.}
Special session `Sparse Data Representations and Applications',  AMS  Southeastern Meeting, Statesboro, 12-13 Mar 2011.
\item {\sl Compressive Sensing insight into the geometry of quasi-Banach spaces.}
Workshop on `Sparse and Low Rank Approximation', Banff, Canada, 6-11 Mar 2011.
\item {\sl Hard Thresholding Pursuit: an algorithm for Compressive Sensing.}
Workshop on `Wavelet and \mbox{Multiscale} Methods', Oberwolfach, Germany, 1-6 Aug 2010.
\item {\sl The Gelfand widths of $\ell_p$-balls for $0<p\leq 1$.}
Minisymposium `Sparse approximation', 7th \mbox{International} Conference on Curves and Surfaces, Avignon, France, 24-30 Jun 2010.
\item {\sl Best sufficient conditions for sparse recovery.}
Minisymposium `Compressive Sensing', 13th \mbox{International} Conference on Approximation Theory, San Antonio, 7-10 Mar  2010.
\item {\sl Reconstructions parcimonieuses: r\'eelle contre complexe.} Journ\'ee `Approximation et Mod\'elisation G\'eom\'etrique' du groupe SMAI--AFA,
Paris, France, 13 Nov 2009.
\item {\sl Minimisation $\ell_1$ et Compressive Sensing.} 9th Mathias Seminar, Cannes, France, 15-16 Oct 2009.
\item {\sl Sparse recovery via $\ell_q$-minimization for $0<q \le 1$.} 
Special session `Sparse approximation and high-dimensional geometry',
8th International Conference on Sampling Theory and Applications, Marseille, France, 18-22 May 2009.
\item {\sl Best conditioned bases in connection with minimal projections.}
Minisymposium `Minimal projections', 12th International Conference on Approximation Theory, San Antonio, 4-8 Mar 2007.
\eitemize

\subsection{Contributed Conference Presentations}
\bitemize
\item {\sl Recovery from corrupted data: recent results for various models},
9th workshop on High Dimensional Approximation (HDA),  Canberra, Australia,  20-24 Feb 2023.
\item {\sl Refined approximability models in optimal recovery under uncertainty},
Signal, Image, Data, Algorithmic Geometry, Modeling, Approximation (SIGMA) workshop,
Marseille, France, 30 Mar-3 Apr 2020. {\small Canceled due to coronavirus.}
\item {\sl Iterative hard thresholding for low-rank recovery from rank-one projections}, Signal Processing with Adaptive Sparse Structured Representations (SPARS) workshop, Toulouse, France 1-4 Jul 2019.
\item {\sl Determining projection constants of univariate polynomial spaces},
16th International Conference on Approximation Theory, Nashville, 19-22 May 2019.
\item {\sl Quikr \& WGSQuikr: Rapid bacterial community reconstruction via compressive sensing.} 
\mbox{Workshop} `Recent Computational Advances in Metagenomics',
13th European Conference on Computational \mbox{Biology},
Strasbourg, France, 6-10 Sep 2014.
\item {\sl On the value of the max-norm of the orthogonal spline projection.} Constructive Theory of Functions, Varna, Bulgaria, 1-7 Jun 2005.
\item {\sl On the least condition number of a basis of quadratic polynomials.} Advances in Constructive Approximation, Nashville, 14-17 May 2003.
\eitemize


\subsection{Seminars}
\bitemize
\item {\sl A Trustworthy Learning Theory? The View from Optimal Recovery}.
RWTH Aachen University, 22 Jun 2023.
\item {\sl Nonstatistical learning theory: the view from optimal recovery.}
Sydney Mathematical Research Institute seminar,
University of Sydney, Australia, 2 Mar 2023.
\item {\sl Three uses of semidefinite programming in approximation theory}.
One World Mathematics of INformation, Data, and Signals (1W-MINDS) seminar,
inter-institutional and online, 12 Jan 2023.
\item {\sl Three uses of semidefinite programming in approximation theory}.
 Data Science seminar,
 Institute for Mathematics and its Applications,
 University of Minnesota, Minneapolis, 8 Nov 2022.
\item {\sl Standard,  one-bit, and saturated compressive sensing} and {\sl Integrating observation errors in optimal recovery}.
Center for Nonlinear Studies, Los Alamos National Laboratory, 23--25 May 2022.
\item {\sl Integrating observation errors in optimal recovery}.  Codes and Expansions (CodEx) seminar,
pan-university and remote, 5 Apr 2022.
\item {\sl Integrating observation errors in optimal recovery}.
Combinatorics  and Probability seminar, 
University of California, Irvine, 9 Mar 2022.
\item {\sl Optimal recovery in the age of data science.}
Computational and Applied Mathematics seminar,
University of Tennessee, Knoxville (remotely), 24 Feb 2021.
\item {\sl Optimal recovery under approximability models, with applications.}
 Data Science seminar,
 Institute for Mathematics and its Applications,
 University of Minnesota, Minneapolis, 17 Sep 2019.
\item {\sl Sparse recovery techniques in metagenomics.}
Computation and Informatics in Biology and Medicine (CIBM) seminar,
University of Wisconsin, Madison, 29 Jan 2019.
\item {\sl Optimal recovery under approximability models, with applications.}
Systems, Information, Learning, and Optimization (SILO) seminar,
Wisconsin Institute for Discovery, Madison, 23 Jan 2019.
\item {\sl Standard, one-bit, and saturated Compressive Sensing.}
Department of Industrial and Systems Engineering, Texas A\&M University, 14 Sep 2018.
\item {\sl Semidefinite programming in approximation theory: two examples.}
RWTH Aachen University, 17 Jul 2018.
\item {\sl Semidefinite programming in approximation theory: two examples.} Multidisciplinary Optimization Seminar in Toulouse, France, 28 May 2018.
\item {\sl Assimilating data to optimally compute quantities of interest.} Alan Turing Institute, London, U.K., 23 Mar 2018.
\item {\sl The usefulness of a modified restricted isometry property.} University of Oxford, U.K., 22 Mar 2018.
\item {\sl Optimal estimation and computation from data.} University of Maryland, 7 Nov 2017.
\item {\sl Computing a quantity of interest from observational data.} CUNY-Courant symbolic-numeric computing seminar, 19 Oct 2017.
\item {\sl The usefulness of a modified restricted isometry property.} Department of Electrical and Computer Engineering, Iowa State University, 25 Jul 2017.
\item {\sl Sparse recovery from binary or saturated measurements.} Department of Statistics and Biostatistics, Rutgers University, 28 Sep 2016.
\item {\sl Some extra structures in sparse recovery.} Department of Electrical and Computer Engineering, Texas A\&M University, 23 Sep 2015.
\item {\sl Two extra structures in sparse recovery: nonnegativity and disjointedness.} Drexel University, 16 Oct 2014.
\item {\sl Classical and one-bit compressive sensing.} 
%Analysis and Applied Math Seminar, 
Kennesaw State University, 13 Nov 2013.
\item {\sl Iterative algorithms in compressive sensing.} INRIA Rennes, France, 28 Mar 2013,
University of Cambridge, U.K., 19 Mar 2013. 
\item {\sl $\ell_1$-minimizations with Weibull matrices.} Wilks Seminar, Princeton Statistics Laboratory, 7 Dec 2012.
\item {\sl Schumaker's conjecture: do Bernstein operators induce P-matrices?} Drexel University, 9 Mar 2012.
\item {\sl Orthogonal matching pursuits in Compressive Sensing.} University of Bonn, Germany, 24 Nov 2011.
\item {\sl On the dimension of multivariate spline spaces.} Drexel University, 11 Nov 2011.
\item {\sl Compressive Sensing and the Hard Thresholding Pursuit algorithm.} University of Utah, 26 Sep 2011.
\item {\sl Recovering sparse vectors via Hard Thresholding Pursuit.} Johns Hopkins University, 17 Mar 2011.
\item {\sl Geometry of $\ell_1^n$ via Compressive Sensing.} VIGRE Seminar, University of Georgia, 15 Feb 2011.
\item {\sl Compressive Sensing and the Hard Thresholding Pursuit algorithm.\hspace{-1.5mm}}  University of Maryland, 1~Dec~2010.
\item {\sl Some open problems in Approximation Theory.} Drexel University, 29 Oct 2010.
\item {\sl Sparse recoveries via Basis Pursuit and Hard Thresholding Pursuit.} Drexel University, 8 Oct 2010.
\item {\sl Variations around the RIP.} University of Bonn, Germany, 3 Jun 2010.
\item {\sl Basis pursuit with pre-Gaussian random matrices.} 
%S\'eminaire de Probabilit\'es et de Statistiques, 
Universit\'e de Franche--Comt\'e, Besan\c{c}on, France, 26 Apr 2010.
\item {\sl Gelfand widths, pre-Gaussian random matrices, joint sparsity.} Vanderbilt University, 15 Mar 2010.
\item {\sl Randomness in Compressive Sensing.} S\'eminaire Parisien de Statistique, Paris, France, 11 Jan 2010.
\item {\sl Un condens\'e de Compressive Sensing.} Journ\'ee 40 ans du Laboratoire Jacques-Louis Lions, Paris, France, 18 Dec 2009.
\item {\sl Three topics in Compressive Sensing.} University of Cambridge, U.K., 29 Oct 2009.
\item {\sl Compressive sensing via $\ell_q$-minimization for $0 < q \le 1$.} University of Edinburgh, U.K., 22 Oct 2009.
\item {\sl Reconstruction parcimonieuse par minimisation $\ell_q$ avec $0<q \le 1$.} INRIA Rennes, France, 23~Jun~2009.
\item {\sl Sparse recovery via $\ell_q$-minimization for $0<q \le 1$.} Universit\'e Pierre et Marie Curie, Paris, France, 26 May 2009.
\item {\sl Compressed Sensing via nonconvex minimization.} Hausdorff Center, Bonn, Germany, 19 Dec 2008.
\item {\sl Condition numbers of finite-dimensional frames.}
%Applied Math Seminar,
University of Georgia, 11 Oct 2007.
\item {\sl Condition numbers of finite-dimensional frames.} Vanderbilt University,
%Computational Analysis Seminar, 
9 Oct 2007
\item {\sl The orthogonal projector onto splines---ongoing development.}
%Computational Analysis Seminar, 
Vanderbilt University, 19 Sep 2006.
\item  {\sl Best conditioned bases and minimal projections.}
%Numerical Analysis Seminar,
University of Cambridge, U.K., 10 Jun 2004.
\item {\sl Some inheritance properties for Chebyshev-type spaces.}
%Numerical Analysis Seminar,
University of Cambridge, U.K., 20 Feb 2003.
\eitemize

%\subsection{Popular Talks}
%\betaremune
%\item {\sl Compressive Sensing: Making the most of few measurements.}
%Drexel University, Dean's seminar, 20~Apr 2011.
%\eetaremune


\section{Miscellaneous Conferences and Workshops}

\bitemize
\item Workshop on `Applied Harmonic Analysis and Data Science', Oberwolfach, Germany,  21-26 Apr 2024.
\item SQuaRE project `Approximation Theory and Semidefinite Programming', AIM,  8-12 May 2023 (San Jose), 22-26 Mar 2021 (online). With M. Dressler, E. de Klerk, M. Joldes, J. B. Lasserre, Y. Xu.
%{\small Online after several postponements.}
\item `Field of Dreams' conference (organized by the Math Alliance),
Institute for Mathematics and its Applications, Minneapolis,  4-6 Nov 2022.
\item NSF Harnessing the Data Revolution PI meeting,
Alexandria,  26-27 Oct 2022.
\item SIAM Conference on `Mathematics of Data Science',
San Diego, 26-30 Sep 2022. Online attendance.
\item Conference on `Interactions between Uncertainty Quantification and Machine Learning',
Clermont-Ferrand, France, 7-9 Jun 2022.
\item Spring Meeting of the Academic Data Science Alliance, 
Irvine, 7-9 Mar 2022.
\item  Workshop on `Computation and Learning in High Dimensions', Oberwolfach, Germany, 1-7 Aug 2021.
\item Thematic Programme on `Applied Functional Analysis and High-Dimensional Approximation',
\mbox{Erwin} Schroedinger International Institute for Mathematics and Physics, Vienna, Austria, 
19 Apr-28 May~2021.
{\small Canceled due to coronavirus.}
\item `Neural Information Processing Systems' (NeurIPS) conference,
Vancouver, Canada, 9-14 Dec 2019.
\item Programme on `Approximation, Sampling and Compression in Data Science',
Isaac Newton Institute, Cambridge, U.K., 3 Jan-28 Jun 2019.
With a Simons Foundation Fellowship (declined).
\item  Workshop on `Applied Harmonic Analysis and Data Processing', Oberwolfach, Germany, 25-31 Mar 2018.
\item SQuaRE project `Developing the theory of $1$-bit compressive sensing',
AIM, 22-26 Aug~2016, 13-17 Jul~2015 (San Jose), 18-22 Nov~2013 (Palo Alto). With R. Baraniuk, D. Needell, Y. Plan, M. Wooters.
%\item Conference `Asymptotic Geometric Analysis III',
%Euler International Mathematical Institute, 
%Saint-Petersburg, Russia, 17-22 June 2016.
\item Workshop on `Optimization and Parsimonious Modeling', 
Institute for Mathematics and its Applications, Minneapolis, 25-29 Jan 2016.
\item Trimester Program on `Mathematics of Signal Processing', Hausdorff Research Institute, Bonn, Germany,
4 Jan-22 Apr 2016.
%\item Workshop on `Applied Harmonic Analysis and Sparse Representations', Oberwolfach, Germany, 16-22 Aug 2015.  
\item Invited Research Fellow at the Semester Program on `High-Dimensional Approximation', 
Institute for Computational and Experimental Research in Mathematics, Brown University,
8 Sept-5 Dec 2014.
\item 2nd International Workshop on Compressed Sensing Applied to Radar,
Bonn, Germany, 17-19~Sep~2013.
\item Annual Meeting of the Canadian Applied
and Industrial Mathematics Society, Quebec City, 16-20 Jun~2013. 
%\item Workshop on `Multiscale and High-Dimensional Problems', Oberwolfach, Germany, 28 Jul-3 Aug 2013.
\item Workshop on `Structure and Randomness in System Identification and Learning',
Institute for Pure and Applied Mathematics, University of California at Los Angeles, 15-18 Jan 2013.
\item DTRA/NSF/NGA Algorithm Workshop, San Diego, 26-29 Nov 2012.
\item Workshop on `Applied Harmonic Analysis and Sparse Approximation', Oberwolfach, Germany, 10-16 Jun 2012.
\item Long Program on `Mathematical and Computational Approaches in High-Throughput Genomics', 
Institute for Pure and Applied Mathematics, University of California at Los Angeles.
Attending the workshop for the period 12 Sep-10 Oct 2011.
\item Concentration week on `Greedy Algorithms in Banach Spaces and Compressed Sensing', Texas A\&M University, 18-22 Jul 2011.
\item `Foundations of Computational Mathematics' conference, Budapest, 4-14 Jul 2011.
\item Trimester Program on `Analysis and Numerics for High-Dimensional Problems', Hausdorff Research Institute, Bonn, Germany.
Attending the workshops for the period 19 Jun-2 Jul 2011.
\item `February Fourier Talks' conference, University of Maryland, 17-18 Feb 2011.
\item Workshop on `High Dimensional Problems and Solutions', Paris, France, 21-22 Jun 2010.
\item Workshop on `Sparsity and Computation', Bonn, Germany, 7-11 Jun 2010.
\item Workshop on `Probability and Geometry in High Dimensions', Marne-la-Vall\'ee, France, 17-21 May 2010.
\item Fall School on `Interactions between Compressed Sensing, Random Matrices, and High Dimensional Geometry', Marne-la-Vall\'ee, France, 16-20 Nov 2009.
\item Summer School on `Theoretical Foundations and Numerical Methods for Sparse Recovery', Linz, Austria, 31 Aug-4 Sep 2009.
\item Workshop on `Nonlinear  Approximation Techniques Using  $L_1$', Texas A\&M University, 16-18~May~2008.
\item 10th SIAM Conference on Geometric Design and Computing, San Antonio, 4-8 Nov 2007.
\item 6th International Conference on Curves and Surfaces, Avignon, France, 29 Jun-5 Jul  2006.
\eitemize


\section{Teaching}

\bitemize
\item {\bf Texas A\&M University} (2015-).
{\sl Graduate courses:} 
Compressive Sensing,
Topics in Mathematical Data Science,
Foundations and Methods of Approximation,
Mathematical Foundations for Data Science.
{\sl Undergraduate courses:} (Honors) Linear Algebra, Advanced Calculus I.
\item {\bf University of Georgia} (2013-15).
{\sl Graduate courses:}
Compressive Sensing,
%Matrix Analysis (minicourse --- TBC).
{\sl Undergraduate courses:}
Calculus I for Science and Engineering,
Calculus II for Science and Engineering.
\item {\bf Drexel University} (2010-13).
{\sl Graduate courses:}
Linear Algebra and Matrix Analysis, 
%(Math504, graduate, Fall 2010)
Approximation Theory,
%(Math680, graduate, Spring 2012)
Compressed Sensing,
%(Math680, graduate, Spring 2011),
Mathematics of Genome Analysis.
%(Math680, graduate, Spring 2012)
{\sl Undergraduate courses: }
Problem Solving for Math Competitions,
%(Math235, Fall 2012)
Probability and Statistics~II,
%(Math312, Spring 2012),
Numerical Analysis~II,
%(Math301, Winter 2012),
Linear Algebra,
%(Math201, Winter 2012),
Calculus~I. 
%(Math121, Winter 2011)
\item {\bf Vanderbilt University} (2006-09). 
{\sl Graduate courses:}
Compressed Sensing. 
%(Math394, graduate, Spring 2009),
{\sl Undergraduate courses:}
Introduction to Numerical Mathematics,
%(Math226/CS255, Spring 2007), 
Methods of Ordinary Differential Equations,
%(Math198, Spring 2008),
 %first- and second-year 
 Calculus~I~\&~III.
 %(Math150A, Fall 2006, Spring 2007, and Fall 2007; Math170, Fall 2008)
\item {\bf University of Cambridge}, U.K. (2003-05). Gave supervisions in Differential
 Equations, Probability, Numbers and Sets, Dynamics, Numerical
 Analysis.
% 2000 (July), Technical University, Ostrava, Czech Republic: Computed a waste
 %management program for a thermal power station.\\
 \item {\bf Ecole Nationale de Commerce}, Paris, France (1999-00). Oral
 examiner in Mathematics, preparing students for the
 entrance examinations to the economic Grandes Ecoles.
 %1999 (July-August), Melbourne, Australia: Practical work experience
 %at Lightning International Pty Ltd (manufacturer of cloth-cutting
 %machines).
 \eitemize
 
  
\section{Advisees}

\begin{tabular}{ll}
{\bf Postdocs} 
& Josiah Park (Aug 2020-Dec 2022), now postdoc at Berkeley National Lab\\
& Richard  G.  Lynch (Aug 2016-Jun 2019), then Instr. Assi. Prof. at Texas A\&M\\
& Jean-Luc Bouchot (Nov 2012-Aug 2014),
then Assi. Prof. at BIT, China\\
%now postdoc at RWTH Aachen\\
& David Koslicki (Jan-Sep 2012), now 
Asso. Prof. at Penn State\\
%& detached at  the Mathematical Biosciences Institute of the Ohio State University\\
{\bf PhD students}  
& Thomas Winckelman (Aug 2023-)\\
%& Tushar Pandey (Aug 2020-) \\
& Chunyang Liao (Aug 2019-May 2023), then postdoc at UCLA\\
& Ryan Malthaner (Aug 2018-Aug 2021)\\
& Bolong Ma (Aug 2017-Aug 2021 AWOL)\\
& Srinivas Subramanian (Aug 2016-Oct 2023) \\
& Mahmood Ettehad (Aug 2016-Jul 2020), then postdoc at the IMA\\
& Michael Minner (Sep 2012-Mar 2016), now at Sandia National Lab\\
{\bf Graduate RAs} & Tom Needham (Summer 2014), now Assi. Prof. at Florida State\\
& Vladlena Powers (Jan 2014-Jul 2014), then PhD student at Columbia\\
& Anchit Agarwal (Aug 2013-Jul 2014)\\
{\bf Undergrad RAs} & Chase Colbert (May-Jun 2020), then Master's student at Texas A\&M
\end{tabular}


\section{Professional Services}

\subsection{Book Series Editor}

\bitemize
\item
{\em Numerical Mathematics and Scientific Computation}, Oxford University Press (Jul 2023- )
\eitemize


\subsection{Journal Editorial Boards}

\bitemize
\item
{\sl Journal of Approximation Theory} (Aug 2017- )
\item
{\sl Sampling Theory, Signal Processing, and Data Analysis} (Jun 2020- )
\item
{\sl Journal of Numerical Mathematics} (May 2021- )
\item 
{\sl Journal of Complexity} (Sep 2023- )
\item 
{\sl Surveys in Approximation Theory} (Sep 2023- )
\eitemize

%\subsection{Technical Program Committee}
%
%\bitemize
%\item
%{\sl SampTA 2019}
%\eitemize
 
\subsection{Reviewing}
\bitemize
\item Refereed for 
%the journals
{\sl Numerical Algorithms} (2023),
{\sl Numerische Mathematik} (2023),
{\sl Constructive Approximation} (2022; 2021; 2018; 2015; 2012; twice in 2010),
{\sl Journal of Functional Analysis} (2022; 2021; 2016),
{\sl Journal of Approximation Theory} (2022; 2017; 2016; 2014; 2012; 2007),
{\sl Comptes Rendus Math\'ematique} (2022; 2020),
{\sl SIAM Journal on Applied Algebra and Geometry} (2022; 2019),
{\sl Linear Algebra and its Applications} (2022; 2012),
{\sl SIAM Journal on Optimization} (2022; 2015),
{\sl Applied and Computational Harmonic Analysis} (2022; 2021; 2019; twice in 2010),
{\sl SIAM Journal on Scientific Computing} (2021; 2020; 2015),
{\sl Optimization Letters} (2021),
{\sl Discrete \& Computational Geometry} (2021),
{\sl EURASIP Journal on Advances in Signal Processing} (2021; 2011),
{\sl Bulletin of the Australian Mathematical Society} (2021),
{\sl IMA Journal of Numerical Analysis} (2020),
{\sl SIAM/ASA Journal on Uncertainty Quantification} (2020),
{\sl Operators and Matrices} (2020),
{\sl SIAM Journal on Mathematics of Data Science} (2019),
{\sl IEEE Transactions on Information Theory} (2019; 2018; 2013; twice in 2011; 2009; 2008),
{\sl Bolet\'in de la Sociedad Matem\'atica Mexicana} (2019),
{\sl Journal of Fourier Analysis and Applications} (2019),
{\sl Annals of Applied Probability} (2019),
{\sl Journal of Machine Learning Research} (2018),
{\sl Journal of Mathematical Analysis and Applications} (2017),
{\sl Advances in Computational Mathematics} (2017; 2013),
{\sl Michigan Mathematical Journal} (2017),
{\sl Monatshefte f\"ur Mathematik} (2016),
{\sl Information and Inference} (2016),
{\sl Inverse Problems} (2015),
{\sl Journal of Theoretical Biology} (2015),
{\sl SIAM Journal on Imaging Sciences} (2014),
{\sl Journal of Algebra} (2014),
{\sl IEEE Signal Processing Letters} (2014; 2013; 2009),
{\sl Digital Signal Processing} (2014),
{\sl \mbox{Foundations} of Computational Mathematics} (2013),
{\sl IEEE Transactions on Signal Processing} (twice in 2013; 2012), 
{\sl Statistics and Probability Letters} (2013),
%{\sl Proceedings of the International Conference on Approximation Theory} (2013; 2010),
{\sl Mathematics of Computation} (2012),
{\sl SIAM Journal on Matrix Analysis and Applications} (2011),
{\sl International Journal of Mathematics and Mathematical Sciences} (2011),
{\sl Inverse Problems and Imaging} (2011),
{\sl Signal Processing} (2011),
and
{\sl IEEE Journal of Selected Topics in Signal Processing} (twice in 2009)
\item Various conferences 
(member of the technical program committee of SampTA~2019;
refereed for NeurIPS~2020,
 iTWIST~2018, COLT 2018, SampTA~2017, AT~2016, SPARS~2015, SampTA~2015, ISIT~2015, CSA~2013, AT~2013, GRETSI 2013, SPARS~2013, SampTA~2013, CAMSAP~2011, AT~2010)
\item Refereed 
books for Cambridge University Press (2020, 2018, 2016),
a book proposal for SIAM (2017),
and a research monograph for the Soci\'{e}t\'{e} Math\'{e}matique de France (2011)
%for the book series {\sl Panoramas et Synth\`eses} (Soci\'{e}t\'{e} Math\'{e}matique de France, 2011)
\item Refereed for the 
{\sl Deutsche Forschungsgemeinschaft} (German equivalent of NSF, 2023, 2022),
the
{\sl National Science Foundation} (2021,2019, 2017),
the
{\sl Israel Science Foundation} (2020),
the
{\sl Natural Sciences and Engineering Research Council of Canada} (2019),
the
{\sl Nederlandse Organisatie voor Wetenschappelijk Onderzoek} (Dutch equivalent of NSF, 2019),
the
{\sl Research Grants Council of Hong Kong} (2018, 2017, 2016, 2015),
the 
{\sl Fonds zur F\"orderung der wissenschaftlichen Forschung} (Austrian equivalent of NSF, 2013) 
and the
 {\sl Agence Nationale de la Recherche} (French equivalent of NSF, 2010)
\item Reviewer for {\sl Mathematical Reviews} (wrote about 60 reviews since 2005)
 %and {\sl Zentralblatt MATH} (2008-present).
 \eitemize
 
 \vspace{0mm}
 
\subsection{Administrative Activities}
\bitemize
\item {\sl Department committees}:  
{\sl TAMU} -- promotion and tenure (2023-), 
graduate (2019-22), executive (2017-19), postdoc (2015-17);
{\sl UGA} -- personnel (2014-15), Cantrell lectures (2014); 
%qualifying exam subcommitee (NA)
{\sl Drexel} -- 
graduate program (qualifying exam subcommittee 2010-13);
tenure-track faculty hiring (2012-13); 
candidacy exams (three occurrences in Sept 2012);
web page (2011-12);
visiting faculty hiring (2010-11).
\item {\sl College and University committees}: 
%{\sl TAMU} -- College of Science webpage commitee (2016-17);
{\sl TAMU} -- 
member of the Faculty Advisory Council for the College of Science (2020-22),
member of the Faculty Advisory Committee for the Texas A\&M Institute of Data Science (2020-21);
{\sl Drexel} -- 
Task force on the future of computing at Drexel (2013);
NSF graduate research fellowship program review (2011-13);
panelist at the meeting on higher education in the U.K. organized by Drexel Study Abroad (May 2012); 
U.K. scholarship review  (Marshall and Gates--Cambridge scholarships, 2011-2012).
\eitemize 
 

\subsection{Organization}

\begin{tabular}{ll}
2023\phantom{-??} & Co-organizer of the inaugural CAMDA conference\\
& College Station, 22-25 May\\
2023 & Co-organizer of the minisymposium {\sl Data Reduction, Approximation, and Computation}\\
& International Conference on Approximation Theory and Beyond, Nashville,  15-18 May\\
%11-14 May\\
%& {\small Postponed from 2020 due to coronavirus.}\\
& {\small Originally scheduled in 2020}\\
2022 & Organizer of the special session {\em Mathematics of Data Science}\\
& Conference on Advances in Data Science, College Station, 21-22 Oct \\
%2019- & Co-organizer of the  {\sl Inverse Problems and Machine Learning Seminar}\\
%& Texas A\&M University\\
2019 & Co-organizer of the week on {\sl Randomness and Determinism in Compressive Data Acquisition}\\
& Workshop in Analysis and Probability, Texas A\&M University,  22-26 July\\
2019 & Co-organizer of the minisymposium {\sl Neural Network Approximation}\\
& 16th International Conference on Approximation Theory, Nashville, 19-22 May\\
2018 & Coordinator of the SQuaRE project {\sl Approximation Theory and Semidefinite Programming}\\
& American Institute of Mathematics, San Jose, one week per year for three years\\
2017 & Co-organizer of the minisymposium {\sl Compressed Sensing and Matrix Completion}\\
& 
21st Meeting of the International Linear Algebra Society,
Iowa State University, 24-28 Jul\\
2016 & Organizer of the minisymposium {\sl Reconstruction Parcimonieuse (Compressive Sensing)}\\
& 43rd Congr\`{e}s National d'Analyse Num\'{e}rique (CANUM), Obernai, France, 9-13 May\\
2015-21 & Organizer of the reading seminar {\sl Data Science and Compressive Sensing}\\
% formerly {\sl Compressive Sensing, Extensions, and Applications}\\
& Texas A\&M University
\end{tabular}

\begin{tabular}{ll}
2013-15 & Coordinator of the {\sl Applied Math Seminar}\\
& University of Georgia\\
2013 & Organizer of the minisymposium {\sl Compressive Sensing}\\
& 14th International Conference on Approximation Theory, San Antonio, 7-10 Apr\\
2011-13 & Organizer of the seminar {\sl Compressive Sensing, Extensions, and Applications}\\
& Drexel University\\
2010 & Organizer of the minisymposium {\sl Compressive Sensing}\\
& 13th International Conference on Approximation Theory, San Antonio, 7-10 Mar\\
2007-09 & Coordinator of the {\sl Computational Analysis Seminar}\\
& Vanderbilt University\\
2008 & Co-organizer of the Shanks Workshop {\sl Nonlinear Models in Sampling Theory}\\
& Vanderbilt University\\
2007 & Co-organizer of the Shanks Workshop {\sl An Advanced Tutorial in Compressed Sensing}\\
& Vanderbilt University\\
2007 &  Co-organizer of the 10th SIAM Conference on Geometric Design and Computing\\
& San Antonio, 4-8 Nov
\end{tabular}



\section{Membership of Associations}

\bitemize
\item Life Member, American Mathematical Society
%\item Member, Mathematical Association of America
\item Member, Society for Industrial and Applied Mathematics
\item Member, Soci\'{e}t\'{e} Math\'{e}matique de France
\item Member, Soci\'et\'e de Math\'ematiques Appliqu\'ees et Industrielles
\item Member, European Mathematical Society
\eitemize


\section{Additional Information}

\subsubsection{Computer Skills} 
MATLAB, Python, Mathematica, Maple, R, Html, JavaScript.

\subsubsection{Languages}
French (native), German (basic), and Spanish (basic).

\subsubsection{Miscellaneous Interests} 
\bitemize
%Long-distance Running and Climbing at a recreational level.\\
\item Team Handball: competition at pre-national and national levels in France and England.
\item Gymnastics:  trained at Forbach Academy (France, 1986-1990); former member of the Cambridge University Team (selected for the Varsity matches against Oxford, 2001 to 2005, winner in 2004).
\item Trampolining: former member of the Cambridge University Team (selected for the Varsity matches against Oxford, 2001 and 2002).
%Gastronomy, Oenology, Art of Bonsai at modest levels
\eitemize

%\subsubsection*{Personal Information}
% Birth: 14 November 1977, in Colmar, France.
% Nationality: French.
% Marital Status: Married.


\end{document}