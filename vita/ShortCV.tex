\documentclass[11pt]{article}
\usepackage{cv_short}
\name{Simon Foucart}
\info{Address: & Texas A\&M University\\
						& Department of Mathematics\\
                          %& 269 Korman Center, 3141 Chestnut Street\\
                          & College Station, TX 77843-3368\\ 
      %Phone: & +1 706-542-2570 \\
      Emails: & foucart@tamu.edu\\
     	          & simon.foucart@centraliens.net\\
      WWW: & \url{http://foucart.github.io}}

\begin{document}
\maketitle
 
\vspace{10mm}
%\section{Current and Past Positions}
%\begin{tabular}{ll}
%2018 & Visiting Researcher (Jun), LAAS-CNRS, Toulouse, France\\
%2017 & Visiting Researcher (Dec), Hong Kong University of Science and Technology, Hong Kong\\
%2015- & Associate Professor of Mathematics, Texas A\&M University, College Station\\
%2015 & Visiting Researcher (May-Jun), University of South Florida, Tampa\\
%2013-15 & Assistant Professor of Mathematics, University of Georgia, Athens\\
%2010-13 & Assistant Professor of Mathematics, Drexel University, Philadelphia\\
%2009-10 & Postdoctoral Researcher, Universit\'{e} Pierre et Marie Curie, Paris, France\\
%& (Laboratoire Jacques-Louis Lions; Mentor: Albert Cohen)\\
%2009 & Visiting Researcher (Jul-Aug), University of Bonn, Germany\\ 
%& (Hausdorff Center for Mathematics; Host: Holger Rauhut)\\
%2006-09 & Assistant Professor of Mathematics (NTT), 
%Vanderbilt University, Nashville\\ 
%& (Center for Constructive Approximation; Mentor: Larry Schumaker)
%\end{tabular}

\vspace{-8mm}

%\section{Current and Past Positions}
\section{Primary Positions}
\begin{tabular}{ll}

2019- & Professor of Mathematics, Texas A\&M University, College Station\\
2015-19 & Associate Professor of Mathematics, Texas A\&M University, College Station\\
2013-15 & Assistant Professor of Mathematics, University of Georgia, Athens\\
2010-13 & Assistant Professor of Mathematics,  Drexel University, Philadelphia\\
2009-10 & Postdoctoral Researcher,  Universit\'{e} Pierre et Marie Curie, Paris, France\\
%& (Laboratoire Jacques-Louis Lions; Mentor: Albert Cohen)\\
2006-09 & Assistant Professor of Mathematics (NTT),  Vanderbilt University, Nashville
% & (Center for Constructive Approximation; Mentor: Larry Schumaker)
\end{tabular}


\section{Secondary Commitments} %Engagements}
\begin{tabular}{ll}
2025-  & Joint Appointee,  Theoretical Division, Los Alamos National Laboratory\\
2021-  & Associate Director for External Academic Engagement, Texas A\&M Institute of Data Science
%2022-24 & CAMDA Director\\
%& \quad Center for Approximation and Mathematical Data Analytics
\end{tabular}


\section{Visiting Positions}
Visits of one month or more at
Clare Hall (University of Cambridge, 2025),
Isaac Newton Institute (University of Cambridge, 2024),
Los Alamos National Laboratory (2023),
Wisconsin Institute for Discovery (UW--Madison, 2019),
LAAS-CNRS (Toulouse, 2018),
Hong Kong University of Science and Technology (2017),
University of South Florida (2015),
University of Bonn (2009).


%\begin{tabular}{ll}
%2025 & Visiting Fellow (May-Jul), Clare Hall, University of Cambridge, U.K.\\ 
%2024 & Visiting Researcher (Jul), Isaac Newton Institute, Cambridge, U.K.\\
%2023 & Visiting Researcher (Jul), Los Alamos National Laboratory\\
%%2019 & Visiting Researcher (Jun-Jul), LAAS-CNRS, Toulouse, France\\
%2019 & Visiting Researcher (Jan-May), Wisconsin Institute for Discovery, UW--Madison\\ 
%2018 & Visiting Researcher (Jun), LAAS-CNRS, Toulouse, France\\
%2017 & Visiting Researcher (Dec), Hong Kong University of Science and Technology, Hong Kong\\
%2015 & Visiting Researcher (May-Jun), University of South Florida, Tampa\\
%2009 & Visiting Researcher (Jul-Aug), University of Bonn, Germany
%\end{tabular}
 

%\section{Higher Education}
%\section{Academic Training}
%\begin{tabular}{lll}
%2001-05 & PhD in Mathematics & University of Cambridge, U.K., Numerical Analysis Group\\
%& Advisor: Alexei Shadrin &\\
%2000-01 & Part III of Math Tripos & University of Cambridge, U.K.\\
%& With distinction & \\ 
%1998-01 & Masters of Engineering & Ecole Centrale Paris, France\\   
%1998-99 & Licence de Math\'{e}matiques & Universit\'{e} Pierre et Marie Curie, Paris, France\\
%\end{tabular}  


\section{Academic Training}
\begin{tabular}{lll}
2001-05 & PhD in Math (Numerical Analysis group) & University of Cambridge, U.K.\\
2000-01 & Part III of Math Tripos (with distinction) & University of Cambridge, U.K.\\
1998-01 & Masters of Engineering & Ecole Centrale Paris, France\\   
1998-99 & Licence de Math\'{e}matiques & Universit\'{e} Pierre et Marie Curie,  Paris, France\\
\end{tabular}  


\section{Honors and Awards}

\begin{tabular}{ll}
2024 & %Recipient of a 
{\sl Heilbronn Distinguished Visiting Fellow}, Isaac Newton Institute, Cambridge, U.K.\\
2019 & %Recipient of a 
{\sl Presidential Impact Fellow}, Texas A\&M University\\
2012 & %Recipient of the  
{\sl Antelo Devereux Award for Young Faculty}, Drexel University\\
2010 & {\sl Journal of Complexity} Best Paper Award\\
%`The Gelfand widths of $\ell_p$-balls for $0<p \le 1$' with A. Pajor, H. Rauhut, T. Ullrich.
%2000-04 & Various scholarships received at the University of Cambridge\\
%& (Dept  of Applied Math and Theoretical Physics; Trinity Hall; Cambridge European Trust)\\
%2001 & Scholar of Trinity Hall, %University of Cambridge, 
%added to the College Register
\end{tabular}
%Travel Grants, University of Cambridge, 2003 and 2005.\\
% Scholarships received at the University of Cambridge
% from the Department  of Applied Mathematics and Theoretical
% Physics (2001-2004),
% from Trinity Hall (2001-2002), and 
%from the Cambridge European Trust 
% (2000-2001 and 2001-2004)

\section{Research Interests}

Mathematics of Data Science and Artificial Intelligence;
Compressive Sensing;
Approximation Theory; 
Computational Mathematics;
Bioinformatics 

%Compressive Sensing;
%Approximation Theory (especially Spline Functions and Minimal Projections); 
%Data Science;
%Computational Mathematics;
%Bioinformatics 
%Abstract and Classical Analysis

 
%\section{Research Grants}
%\section{External Funding}

%`Improving Analysis of Microbial Mixtures through Sparse
%Reconstruction and Statistical Inference'\\
%\hspace{18mm}PI; coPIs: Gail Rosen (Drexel Engineering) and Loni Philip Tabb (Drexel Biostatistics)\\
%\hspace{18mm}NSF grant DMS-1120622, \$666,322, Sep 2011-Aug 2014 (extended until Aug 2015)

%\begin{tabular}{lll}
%%2023-28 & NSF; senior personal,  PI: S.  Banerjee (TAMU Chemistry); \$30,000,000\\
%%& \hspace{-15mm} 
%%{\small STC:} Center for Mathematical, Molecular, and Materials Foundations of Complementary Intelligence\\
%2023 & NSF; co-PI,  PI: S. Wojtowytsch (TAMU Math); \$35,200\\
%& \hspace{-15mm}
%{\small Conference:} Inaugural CAMDA Conference\\
%2021-24 & NSF; sole PI; \$149,783\\
%& \hspace{-15mm}
%{\small CDS\&E-MSS:} Optimal Recovery in the age of Data Science\\
%2020-25 & ONR; local CoPI, local PI: R.~DeVore (TAMU Math); \$883,622;
%Lead: Rice University\\
%& \hspace{-15mm}
%{\small MURI:} Theoretical foundations of Deep Learning \\
%2019-22 & NSF; senior personnel (executive committee), PI: B.~Mallick (TAMU  Statistics);  \$1,416,522 \\
%& \hspace{-15mm}
%{\small TRIPODS:} Texas A\&M Research Institute for Foundations of Interdisciplinary Data Science\\
%2018-21 & NSF; coPI, PI: D.~Koslicki (Oregon State Math), coPI: I.~Ivanov (TAMU Vet Med); 
%\$292,041 \\
%& \hspace{-15mm}
%{\small QuBBD:} Fast, efficient mathematical approach to the analysis of the
%human microbiome through\\ 
%& \hspace{-15mm} biodiversity optimization\\
%2016-19 & NSF; sole PI;
%\$99,535 
%%;\$257,130
%\\
%& \hspace{-15mm}
%{\small CDS\&E-MSS:} Recovery of high-dimensional structured functions
%\\
%2011-15 & NSF; PI, coPIs: G. Rosen (Drexel Engineering), L. P. Tabb (Drexel Biostatistics); 
%\$666,322\\
%%grant DMS-1120622\\
%& \hspace{-15mm}
%{\small ATD:} Improving analysis of microbial mixtures through sparse
%reconstruction and statistical inference
%\end{tabular}

%\section{Internal Funding}

%\begin{tabular}{lll}
%2023 & Texas A\&M ASCEND Initiative; coPI, PI: R. Arroyave (Engineering), \$464K\\
%& \hspace{-15mm}
%%{\small TPT:} Autonomous Scientific Discovery
%{\small TPT:} Foundations of Autonomous Materials Discovery\\
%2023 & Texas A\&M University System National Laboratories Office, \$32K\\
%& \hspace{-15mm}
%Development fellowship to strengthen collaborations with Los Alamos National Laboratory\\
%2022 & College of Arts and Sciences;
%coPI, PI: T. Logan (Atmospheric Sciences),  \$10K\\
%& \hspace{-15mm}
%{\small Seed Grant Promoting Research Collaborations:}
%Do HLMA data contain evidence of space lightning?\\
%2022 & Texas A\&M University System National Laboratories Office, \$1.5K\\
%& \hspace{-15mm}
%Exploration minigrant to initiate collaborations with Los Alamos National Laboratory\\
%2021 & Texas A\&M Institute of Data Science, \$15K\\
%& \hspace{-15mm}
%TAMIDS Course Development Grant for {\em MATH 664: Topics in Mathematical Data Science}\\
%2021-22 & Texas A\&M; PI,  CoPI: R.~Tuo (Engineering), S.~Shahrampour (now at Northeastern), \$30K\\
%& \hspace{-15mm}
%{\small T3 Triads:} Learning more efficiently with less labels\\
%2019-20 & Texas A\&M; coPI, PI: S.~Shahrampour (Engineering), CoPI: B.~Hanin (Math), \$32,876\\
%& \hspace{-15mm}
%{\small T3 Triads:} Trade-offs between approximation and generalization in learning systems
%\end{tabular}


%\section{Honors and Awards}
%
%\begin{tabular}{ll}
%2019 & Recipient of a {\sl Presidential Impact Fellowship}, Texas A\&M University\\
%2012 & Recipient of the  {\sl Antelo Devereux Award for Young Faculty}, Drexel University\\
%2010 & {\sl Journal of Complexity} Best Paper Award\\
%%`The Gelfand widths of $\ell_p$-balls for $0<p \le 1$' with A. Pajor, H. Rauhut, T. Ullrich.
%2000-04 & Various scholarships received at the University of Cambridge\\
%& (Dept  of Applied Math and Theoretical Physics; Trinity Hall; Cambridge European Trust)\\
%2001 & Scholar of Trinity Hall, %University of Cambridge, 
%added to the College Register
%\end{tabular}
%%Travel Grants, University of Cambridge, 2003 and 2005.\\
%% Scholarships received at the University of Cambridge
%% from the Department  of Applied Mathematics and Theoretical
%% Physics (2001-2004),
%% from Trinity Hall (2001-2002), and 
%%from the Cambridge European Trust 
%% (2000-2001 and 2001-2004)


\section{Publications}

\subsection{Books}
\betaremune
\item{\sl Mathematical Pictures at a Data Science Exhibition.}\\
Cambridge University Press, 2022. 
\item {\sl A Mathematical Introduction to Compressive Sensing.}\\
Birkh\"auser, Applied and Numerical Harmonic Analysis, 2013.  With H. Rauhut.
\eetaremune

%\subsection{Books Edited}
%\betaremune
%\item{\sl Explorations in the Mathematics of Data Science --- the Inaugural Volume of the Center for Approximation and Mathematical Data Analytics.}\\
%Birkh\"auser, Applied and Numerical Harmonic Analysis, 2024. 
%With S. Wojtowytsch.
%\eetaremune

\subsection{Refereed Articles}
Selected from 2 surveys, 59 journals papers (including 1 as submitted), and 12 proceedings papers:
\betaremune
\item {\sl Learning the maximum of a H\"older function from inexact data}.\\
Proceedings of the American Mathematical Society,
154/3, 1077--1091, 2026.
\item {\sl Worst-case learning under a multifidelity model}.\\
SIAM/ASA Journal on Uncertainty Quantification, 13, 171--194, 2025. With N. Hengartner.
\item {\sl Nonlinear approximation and (deep) ReLU networks.}\\
Constructive Approximation,  55, 127--172, 2022.
With I.~Daubechies, R.~DeVore, B.~Hanin, G.~Petrova.
\item {\sl Determining projection constants of univariate polynomial spaces}.\\
Journal of Approximation Theory, 235, 74--91, 2018.
With J. B. Lasserre.
\item {\sl Exponential decay of reconstruction error from binary measurements of sparse signals.}\\
IEEE Transactions on Information Theory, 63/6, 3368--3385, 2017.
With R. Baraniuk, D. Needell, Y.~Plan, and \mbox{M.~Wootters}.
\item {\sl Stability and robustness of $\ell_1$-minimizations with Weibull matrices and redundant dictionaries.}\\
Linear Algebra and its Applications, 441, 4--21, 2014.
\item {\sl Quikr: a method for rapid reconstruction of bacterial communities via compressive sensing.}\\
Bioinformatics, 29/17, 2096--2102, 2013. 
With D. Koslicki, G. Rosen.
\item {\sl Hard thresholding pursuit: an algorithm for Compressive Sensing.}\\
SIAM Journal on Numerical Analysis, 49/6, 2543--2563, 2011.
\item {\sl The Gelfand widths of $\ell_p$-balls for $0<p \le 1$.}\\
Journal of Complexity, 26/6, 629--640, 2010. 
With A. Pajor, H. Rauhut, T. Ullrich.
\item {\sl  Sparsest solutions of underdetermined linear systems via $\ell_q$-minimization for $0 <  q  \le  1$.}\\
Applied and Computational Harmonic Analysis, 26/3, 395--407, 2009. 
With M.-J. Lai.
\eetaremune

	
\section{Oral Presentations}

%\subsection{Popular Talks}
%\betaremune
%\item {\sl Compressive Sensing: Making the most of few measurements.}
%Drexel University, Dean's seminar, 20~Apr 2011.
%\eetaremune

6 plenary addresses at international meetings,
6 short courses in 5 countries,
73 invited conference presentations across the world,
7 contributed conference presentations,
15 colloquia,
53 seminar talks.



\section{Professional Services}

\subsection{Editorial Boards}

\bitemize
\item
{\sl Journal of Approximation Theory} (journal, Aug 2017- )
\item
{\sl Sampling Theory, Signal Processing, and Data Analysis} (journal, Jun 2020- )
\item
{\sl Journal of Numerical Mathematics} (journal, May 2021- )
\item 
{\em Numerical Mathematics and Scientific Computation} (book series, Oxford University Press, Jul 2023-~)
\item
{\sl Journal of Complexity} (journal, Sep 2023- )
\item 
{\sl Surveys in Approximation Theory} (online journal, Sep 2023- )
\eitemize

 
\subsection{Reviewing}

Refereed for 51 journals 
(inc. Advances in Mathematics, 
Foundations of Computational Mathematics, 
Journal of Functional Analysis, 
Proceedings of the AMS,
SIAM/ASA Journal on Uncertainty Quantification,
SIAM Journal on Applied Algebra and Geometry),
 15 conferences, 
 3 book publishers,
and the national science foundations of 8 countries.



\subsection{Organization}

(Co)organized 3 workshops,
8 special sessions at conferences,
and 2 conferences, including the inaugural conference of Texas A\&M's (now defunct) Center for Approximation and Mathematical Data Analytics, of which I was the director.



\end{document}